\singlespacing
\hypersetup{
    colorlinks,%
    citecolor=black,%
    filecolor=black,%
    linkcolor=black,%
    urlcolor=black 
    %urlcolor=mygreylink     % can put red here to better visualize the links
}
\urlstyle{same}
\definecolor{mygrey}{gray}{.85}
\definecolor{mygreylink}{gray}{.40}
\textheight=9.0in
\raggedbottom
\raggedright
\setlength{\tabcolsep}{0in}

% Adjust margins
\addtolength{\oddsidemargin}{-0.375in}
\addtolength{\evensidemargin}{0.375in}
\addtolength{\textwidth}{0.5in}
\addtolength{\topmargin}{-.375in}
\addtolength{\textheight}{0.75in}

%-----------------------------------------------------------
%Custom commands
\newcommand{\resitem}[1]{\item #1 \vspace{-2pt}}
\newcommand{\resheading}[1]{{\large \colorbox{mygrey}{\begin{minipage}{\textwidth}{\textbf{#1 \vphantom{p\^{E}}}}\end{minipage}}}}
\newcommand{\ressubheading}[4]{
\begin{tabular*}{6.5in}{l@{\extracolsep{\fill}}r}
		\textbf{#1} & #2 \\
		\textit{#3} & \textit{#4} \\
\end{tabular*}\vspace{-6pt}}

\newcommand{\ressubsubheading}[2]{
\begin{tabular*}{6.5in}{l@{\extracolsep{\fill}}r}
		\textit{#1} & \textit{#2} \\
\end{tabular*}\vspace{-6pt}}
\renewcommand{\baselinestretch}{0.9}
%-----------------------------------------------------------

\newcommand{\mywebheader}{
\begin{tabular*}{6.5in}{l@{\extracolsep{\fill}}r}
	\textbf{\LARGE Xudong Liu} 
		%& \href{http://www.cs.uky.edu/~liu/}{http://www.cs.uky.edu/$\sim$liu/} $\bullet$ liu@cs.uky.edu\\
		& Born in Zaozhuang, Shandong, China.
	\end{tabular*}
\\
\vspace{0.1in}}

% CHANGE HEADER SOURCE HERE
\mywebheader

%%%%%%%%%%%%%%%%%%%%%%
%\resheading{Research Interests}
%%I am looking for a 2015 summer intern position in software development engineering.
%My research focuses on exciting topics in artificial intelligence and social science, including
%preferences (i.e., preference modeling, learning and reasoning), 
%data-driven and computational social science, 
%knowledge representation and reasoning, 
%decision theory, data mining, and machine learning.
%\vspace{3pt}

%%%%%%%%%%%%%%%%%%%%%%
\resheading{Education}
			\ressubheading{University of Kentucky}{USA}
			{Doctor of Philosophy, computer science
				}
			{Aug. 2010 -- present}\\  \vspace{6pt}
			{GPA:\textbf{4.00}/4.00 \\ Advisor: Dr. Miroslaw Truszczynski}  
				%\vspace{-6pt}
				%{ \small
				%\begin{itemize}
				%	%\resitem{\textbf{Available for internship: July 23, 2009; Graduation follows internship}}
				%	\resitem{\textbf{Courses}: 
				%			Modern Operating Systems, 
				%			Numerical Analysis, 
				%			Algorithm Design,
				%			Distributed Operating System, 
				%			Satisfiability and Equivalence Checking, 
				%			Database Systems, 
				%			Computer Networks, 
				%			Networks Security, 
				%			Preferences in AI and Decision Theory, 
				%			Principles of Constraint Satisfaction,
				%			Comparative Decision Making, 
				%			Preparing Future Faculty, and Grant Writing.}
				%\end{itemize}
				%}
%\begin{comment}
			\ressubheading{Harbin Institute of Technology}{China}
				{Bachelor of Engineering, software engineering}{Aug. 2006 -- July 2010}\\ \vspace{6pt}
				{GPA:3.56/4.00 \\ Advisor: Prof. Yushan Sun}
%\end{comment}
\vspace{3pt}


%%%%%%%%%%%%%%%%%%%%%%
\resheading{Employment}
			\ressubheading{{R\&D Intern}}{Palo Alto Research Center (PARC), USA}
			{Supervisor: Dr. Christian Fritz}
			{Jun. 2015 -- Aug. 2015}
			\vspace{6pt}
				%{ 
				%\begin{itemize}
				%	\resitem{Conducted research on and developed system modules for representing 
				%		and reasoning about user constraints and preferences in 
				%		trip planning.}
				%\end{itemize}
				%}
			\ressubheading{{Graduate Research Assistant}}{University of Kentucky, USA}
			{Advisor: Dr. Miroslaw Truszczynski}
			{Aug. 2010 -- May 2015}
			\vspace{6pt}
				%{
				%\begin{itemize}
				%	\resitem{Conducted research on logic-based knowledge representation formalisms, 
				%		studied and built
				%		tools for representing and reasoning about constraints and preferences in artificial intelligence.}
				%\end{itemize}
				%}
			\ressubheading{{Graduate Teaching Assistant}}{University of Kentucky, USA}
			{Advisors: Dr. Truszczynski, Dr. Pike and Dr. Moore}
			{Aug. 2010 -- May 2015}
			\vspace{6pt}
				%{
				%\begin{itemize}  \itemsep -2pt
				%	\resitem{CS215 - Introduction to Program Design and Problem Solving: leading lab
				%		sessions, grading assignments, preparing exam questions and solutions,
				%		and holding office hours.}
				%	\resitem{CS375 - Logic and Theory of Computing: grading assignments, 
				%		preparing assignment solutions,
				%		and holding office hours.}
				%	\resitem{CS463G - Introduction to Artificial Intelligence: guest instructor,
				%		on knowledge representation and reasoning, e.g., propositional logic, and
				%		first-order logic.}
				%\end{itemize}
				%}
			\ressubheading{{Undergraduate Teaching Assistant}}{Harbin Institute of Technology, China}
			{Advisor: Prof. Yushan Sun}
			{Aug. 2008 -- May 2010}
			\vspace{6pt}
				%{
				%\begin{itemize}  \itemsep -2pt
				%			\resitem{Compilers and J2EE, including
				%				leading lab sessions and grading assignments.}
				%\end{itemize}
				%}
			\ressubheading{{Undergraduate Intern}}{Information Security Lab, Harbin Institute of Technology, China}
			{Advisor: Prof. Yushan Sun}
			{Aug. 2009 -- May 2010}
				%{
				%\begin{itemize}  \itemsep -2pt
				%			\resitem{Used PHP, MySQL and Apache Tomcat to implement modules of a management system.}
				%\end{itemize}
				%}
\vspace{3pt}


%%%%%%%%%%%%%%%%%%%%%%%%%%%%%%%%%%%%%%%%%%%%%%%%%%%%%%
\resheading{Professional Services}
	\begin{description} \itemsep -3pt
		\item[Student member:] {AAAI}
		\item[Student volunteer:] {AAAI-15}
		\item[Program committee:] {IJCAI-16, IJCAI-13}
		\item[Paper reviewer:] {JAIR, AAAI-14, ISAIM-14}
		\item[Local arrangement committee:] {ADT-15, LPNMR-15, ICLP-11, NonMon@30-10}
	\end{description} % End Skills list
\vspace{3pt}



%%%%%%%%%%%%%%%%%%%%%%%%%%%%%%%%%%%%%%%%%%%%%%%%%%%%%%
\resheading{Refereed Publications}
\begin{enumerate}  \itemsep -3pt
	\item \textbf{Xudong Liu}.
		\textit{Modeling, Learning and Reasoning with Qualitative Preferences}.
		In Proceedings of the 4th International Conference on Algorithmic Decision Theory (ADT),
		volume 9346, pages 587-592, 2015. Springer
	\item \textbf{Xudong Liu} and Miroslaw Truszczynski.
		\textit{Reasoning with Preference Trees over Combinatorial Domains}.
		In Proceedings of the 4th International Conference on Algorithmic Decision Theory (ADT),
		volume 9346, pages 19-34, 2015. Springer
	\item \textbf{Xudong Liu} and Miroslaw Truszczynski.
		\textit{Learning Partial Lexicographic Preference Trees over Combinatorial Domains}.
		In Proceedings of the 29th AAAI Conference on Artificial Intelligence (AAAI),
		pages 1539-1545, 2015. AAAI Press
	\item \textbf{Xudong Liu} and Miroslaw Truszczynski.
		\textit{Preference Trees: A Language for Representing and Reasoning about Qualitative Preferences}.
		In Proceedings of the 8th AAAI Multidisciplinary Workshop on Advances in Preference 
		Handling (MPREF), pages 55-60, 2014. AAAI Press
	\item \textbf{Xudong Liu} and Miroslaw Truszczynski.
		\textit{Aggregating Conditionally Lexicographic Preferences Using Answer Set Programming Solvers}.
		In Proceedings of the 3rd International Conference on Algorithmic Decision Theory (ADT),
		volume 8176, pages 244-258, 2013. Springer
	\item Matthew Spradling, Judy Goldsmith, {\bf Xudong Liu}, Chandrima Dadi and Zhiyu Li.
		\textit{Roles and Teams Hedonic Game}.
		In Proceedings of the 3rd International Conference on Algorithmic Decision Theory (ADT), 
		volume 8176, pages 351-362, 2013. Springer
	\item {\bf Xudong Liu}. \textit{Aggregating Lexicographic Preference Trees Using Answer Set Programming: Extended Abstract}. 
In 23rd International Joint Conference on Artificial Intelligence Doctoral Consortium (IJCAI DC), 2013.
\end{enumerate}

\vspace{3pt}
%%%%%%%%%%%%%%%%%%%%%%%%%%%%%%%%%%%%%%%%%%%%%%%%%%%%%%

%\resheading{Technical Talks}
%
%\begin{enumerate}  \itemsep -3pt
%	\item \textit{Personalization in Trip Planning}. Oral.
%		At the Keeping Current Seminar, Department of Computer Science,
%		University of Kentucky, 2015.
%	\item \textit{Preference and Social Choice over Combinatorial Domains}. Oral. 
%		At the 1st ADT/LPNMR Doctoral Consortium (ADT/LPNMR-DC-15),
%		Lexington, Kentucky, USA, 2015.
%	\item \textit{Modeling, Learning and Reasoning with Qualitative Preferences}. Poster. 
%		At the 1st ADT/LPNMR Doctoral Consortium (ADT/LPNMR-DC-15),
%		Lexington, Kentucky, USA, 2015.
%	\item \textit{Reasoning with Preference Trees over Combinatorial Domains}. Oral.
%		At the 4th International Conference on Algorithmic Decision Theory (ADT-15),
%		Lexington, Kentucky, USA, 2015.
%	\item \textit{On Personalizability and Extensibility of Multi-Modal Trip Planning}. Oral.
%		At Dialog, Palo Alto Research Center, Palo Alto, California, USA, 2015.
%	\item \textit{Constraints and Preferences in Multi-modal Trip Planning}. Poster.
%		At the PARC Summer Intern Poster Session, Palo Alto Research Center, 
%		Palo Alto, California, USA, 2015.
%	\item \textit{Learning Partial Lexicographic Preference Trees over Combinatorial Domains}. 
%		Oral (ratio: 12\%=238/1991).
%		At the 29th AAAI Conference on Artificial Intelligence (AAAI-15), 
%		Austin, Texas, USA, 2015.
%	\item \textit{Preference Trees: A Language for Representing and Reasoning about Qualitative Preferences}. Oral.
%		At the 8th Multidisciplinary Workshop on Advances in Preference Handling (MPREF-14), 
%		Quebec City, Canada, 2014.
%	\item \textit{Aggregating Conditionally Lexicographic Preferences Using Answer Set Programming Solvers}. Oral.
%		At the 3rd International Conference on Algorithmic Decision Theory (ADT-13),
%		Universite lebre de Bruxelles, Belgium, 2013.
%	\item \textit{Reasoning About Lexicographic Preferences Over Combinatorial Domains}.
%		At the Keeping Current Seminar, Department of Computer Science, 
%		University of Kentucky, 2013.
%	\item \textit{Roles and Teams Hedonic Game}. Oral.
%		At the 7th Multidisciplinary Workshop on Advances in Preference Handling (MPREF-13),
%		Tsinghua University, Beijing, China, 2013.
%	\item \textit{Aggregating Conditionally Lexicographic Preferences Using Answer Set Programming Solvers}. Oral.
%		At the 7th Multidisciplinary Workshop on Advances in Preference Handling (MPREF-13),
%		Tsinghua University, Beijing, China, 2013.
%	\item \textit{Aggregating Lexicographic Preference Trees Using Answer Set Programming: Extended Abstract}. Poster.
%		At the 23rd International Joint Conference on Artificial Intelligence 
%		Doctoral Consortium (IJCAI-DC-13), Tsinghua University, Beijing, China, 2013.
%	\item \textit{Answer Set Programming using Gringo/Clasp}. Oral.
%		At the Keeping Current Seminar, Department of Computer Science,
%		University of Kentucky, 2011.
%\end{enumerate}
%\vspace{3pt}

%%%%%%%%%%%%%%%%%%%%%%%%%%%%%%%%%%%%%%%%%%%%%%%%%%%%%%
\resheading{Honors and Awards}
{\sl Verizon Fellowship}  \hfill Fall 2015 - Spring 2016\\
{\sl Graduate Teaching Assistantship}  \hfill Fall 2014 - Spring 2015, Fall 2012 - Spring 2013\\
{\sl AAAI-15 Student Volunteer and Scholarship Award}  \hfill Jan. 2015\\
{\sl International Student Tuition Scholarship}  \hfill Jan. 2015\\
{\sl Nominee of the Dissertation Year Fellowship}  \hfill Dec. 2014\\
{\sl Harrison D. Brailsford Graduate Scholarship}  \hfill Oct. 2014\\
{\sl Kentucky Opportunity Fellowship Awards}  \hfill July 2013 - June 2014\\
{\sl Nominee of the ACM Award for Outstanding Teaching Assistant}  \hfill 2013\\
{\sl NSF Student Travel Award}  \hfill Aug. 2013\\
{\sl IJCAI-13 Travel Grant Award}  \hfill Aug. 2013\\
{\sl Graduate Research Assistantship}  \hfill Fall 2010 - Spring 2013\\
{\sl Daniel R. Reedy Quality Achievement Fellowship} \hfill Aug. 2010 - May 2013\\
{\sl UK Student Travel Funding Awards}  \hfill 2013 (IJCAI, ADT), 2014 (AAAI)\\
{\sl Chinese National Endeavor Scholarship} \hfill Fall 2008\\
{\sl Outstanding Student Scholarships} \hfill Fall 2006 -  Spring 2010
\vspace{3pt}

%%%%%%%%%%%%%%%%%%%%%%%%%%%%%%%%%%%%%%%%%%%%%%%%%%%%%%
%\resheading{Technical Skills}
%	\begin{description} \itemsep -3pt
%		\item[Programming languages:] { C/C++,
%			C\#, Java, Matlab, Python, Perl, SQL, PHP, HTML}
%		\item[System and software:] { 
%			Linux/Macintosh/Windows, Qt, Git, Answer Set
%			Programming tools, Microsoft Visual Studio, Eclipse, MySql, Microsoft SQL Server
%		}
%	\end{description} % End Skills list


%%%%%%%%%%%%%%%%%%%%%%%%%%%%%%%%%%%%%%%%%%%%%%%%%%%%%%
%\resheading{Selected Projects}
%
%\begin{description}  \itemsep -3pt
%\item[Preference Tree Learning System:] 
%	{ \footnotesize Given a set of pairwise preferences, called examples, 
%	between combinatorial objects,
%	the system learns a partial lexicographic preference tree, preferably small in size,
%	that are consistent with all (or preferably many) given examples.
%	[C++, matlab]}
%
%\item[Logic and SAT Based Social Choice System:] 
%	{ \footnotesize Given votes expressed as lexicographical preference trees, the system
%	computes either a winner that is an alternative with the highest score according
%	to a voting rule (Borda, k-approval, etc), or an alternative whose score is at
%	least some given threshold value. The two problems are modeled as ASP and WPM
%	programs.
%	[C++, clingo, clingcon, toulbar2; Supported by NSF grant IIS-0913459]}
%
%\item[Voting-based Multi-player Gaming Matching System:]
%	{ \footnotesize Given individual preferences from players on team 
%	compositions and individual strategies within a composition, 
%  the system computes a good match of teams of players that 
%	optimize the players’ preferences to some degree.
%	[C++, python, matlab; Partially supported by NSF grant IIS-0913459]}
%
%\item[Preferences Modeling and Optimization:] 
%	{ \footnotesize A user is asked to provide preferences with importance levels. A preference is
%	expressed as a propositional formula or an answer set optimization preference
%	rule for reasoning in the settings of possibilistic logic, leximin ordering, discrimin
%	ordering and Pareto dominance. Optimal solutions are computed.
%	[C++, flex\&bison, gidl/minisatid, clingo; Supported by NSF grant IIS-0913459]}
%
%\item[QoS IP Routing Decision Making Circuits:] 
%	{ \footnotesize Led a programming team to design and implement a GENI network on top of
%	ProtoGENI, where a source node established a route from itself to a destination
%	node by sending \textit{link state advertisement} packets and using Dijkstra's
%	algorithm to compute the shortest path.
%	[C++, ProtoGENI]}
%
%\item[Database System:] 
%	{ \footnotesize Implemented a database system from scratch that handles transactions such as
%	queries, aggregates, insertion and deletion.
%	[Java, Eclipse]}
%\end{description}
%\vspace{3pt}

%%%%%%%%%%%%%%%%%%%%%%%%%%%%%%%%%%%%%%%%%%%%%%%%%%%%%%
%\resheading{References}
%{\bf Dr. Miroslaw Truszczynski}, Professor\\
%Department of Computer Science,
%University of Kentucky \\
%329 Rose Street,
%Lexington, KY 40506 USA \\
%Office Phone: (859) 257-6738\\
%Email: mirek@cs.engr.uky.edu
%\vspace{0.2cm}
%
%{\bf Dr. Christian Fritz}, Area Manager\\
%System Sciences Lab,
%Palo Alto Research Center \\
%3333 Coyote Hill Road,
%Palo Alto, CA 94304 USA \\
%Office Phone: (650) 812-4876 \\
%Email: cfritz@parc.com
%\vspace{0.2cm}
%
%{\bf Dr. Victor Marek}, Professor\\
%Department of Computer Science,
%University of Kentucky \\
%329 Rose Street,
%Lexington, KY 40506 USA \\
%Office Phone: (859) 257-3496\\
%Email: marek@cs.uky.edu
%\vspace{0.2cm}
%
%{\bf Dr. Yi Pike}, Lecturer\\
%Department of Computer Science,
%University of Kentucky \\
%329 Rose Street,
%Lexington, KY 40506 USA \\
%Office Phone: (859) 218-5769\\
%Email: yipike@cs.uky.edu
