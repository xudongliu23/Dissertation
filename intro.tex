%%%% INTRO FUN! %%%%%%
%% start of comments
%\tc{
%	Here I will talk about the field I am working in, namely, 
%	preferences in Artificial Intelligence and 
%	Social Choice.
%	I will discribe the problems I am interested in in the field and explain
%	why these problems are interesting, important and
%	challenging for researchers to study. I will briefly mention other
%	people's work.
%	I will outline this document at the end of this section.
%}
%% end of comments

Preferences are an essential component in areas such as constraint 
satisfaction, decision making, and social choice theory. 
I focus on problems in preference representation, reasoning and learning.

Preferences can be represented in a quantitative or qualitative manner.
For the former, agents express preferences in a numerical form of
a value function that precisely assesses the degree of satisfaction
of objects
(often called \textit{outcomes} or \textit{alternatives}).
Specifying preferences as value functions on alternatives
is feasible for humans in some situations, e.g., 
when the number of alternatives is limited.
In other circumstances, particularly when the number of alternatives is large, however,
people often cannot express their 
preferences directly and accurately as value 
functions \cite{Domshlak20111037}.
%due to the burden of deciding the value function for
%a large number of alternatives \cite{Domshlak20111037}.

Assume an agent is given three flavors of ice-cream 
$strawberry$, $chocolate$ and $vanilla$, and asked to describe 
her preference among them.
The agent could think of a value function that
assigns quantities (\textit{utilities}) to each outcome
based on a scale from 1 to 10, with 10 representing the most satisfaction.
For instance, the agent could give the following value function:
\begin{center}
	$strawberry \mapsto 6$, $chocolate \mapsto 9$ and $vanilla \mapsto 3$.
\end{center}
This function shows that the favorite alternative to the agent is
$chocolate$ with the highest utility, and
$strawberry$ is preferred over $vanilla$.

Instead of rating alternatives quantitatively, it is
often easier and more intuitive to give preferential information
in small pieces in a qualitative way,
e.g., to specify binary preference relations.
Thus, the same agent could rank the flavors as the following
set of binary comparisons:
\begin{center}
	$\{chocolate \succ vanilla, chocolate \succ strawberry,
	strawberry \succ vanilla\}$.
\end{center}
Note that one can formulate the qualitative preferences from
the value function, but not vice versa.

Several preference formalisms in both categories have been developed.
Numeric preference systems include 
fuzzy constraint satisfaction \cite{343640} \cite{Bisterelli},
generalized additive independence networks (GAI nets) \cite{BacchusG95,LIP61766},
conditionally utility independence networks (CUI nets) \cite{EngelW06},
and expected utility networks (EU nets) \cite{Mura2013}, while 
qualitative preference paradigms proposed in AI community include
penalty logic, 
possibilistic logic, conditional preference networks (\emph{CP-nets}) \cite{Kaci:Pref},
conditional preference networks with tradeoffs (\emph{TCP-nets}) \cite{BrafmanD02:TCP},
lexicographic preference trees (\textit{LP trees}) \cite{booth:learningLP},
conditional preference theories (\textit{CP theories}) \cite{Wilson04extendingcp-nets},
and answer set optimization theories (\textit{ASO theories}) \cite{Brewka:ASO,Brewka04}.
% Each of these formalisms provides the user
% with a concise way to express her preferences. 

Once we fix a preference formalism, say $\cF$, in which preferences of agents
are specified, learning preferences expressed in $\cF$ from agents
becomes a problem that has drawn attention from many AI researchers.
Different techniques have been proposed to learning preferences
in $\cF$
such as active learning (or preference elicitation) and passive learning 
\cite{Furnkranz:pref_learning}.
In the process of
active learning, the algorithm asks the user for a pairwise 
preference relation between two given alternatives and constructs
an instance of $\cF$ as more pairwise relations are elicited.
For passive learning, the algorithm assumes that a set of
pairwise relations are somehow obtained and builds an instance
of $\cF$ with no more information from the user.

Provided with preferences learned and presented in a formalism,
what can we say about the agent's preferences over outcomes?
Reasoning problems can be classified
based on the number of agents $n$ from whom the preferences
are elicited:
\begin{enumerate} \itemsep -4pt
	\item $n=1$: individual decision making,
	\item $n>1$: collaborative decision making.
\end{enumerate}
In case $n=1$, we focus on optimization of the agent's preferences and
help her make a better decision by, for example, computing an optimal
alternative or comparing two given alternatives.  For the case
where $n>1$, it is important to calculate
a collective decision (e.g., a winning outcome or ranking)
among the group of agents.

One of the problems in preference reasoning is to
aggregate preferences of a group of
agents (referred
to as \emph{voters} in social choice), which is central to collective decision making and has been studied
extensively in social choice theory. 
Let us consider such a scenario, where we are
given a set of alternatives $X = \{ a,b,c,d,e \}$ and a set $P_X$ of 10 
preferences (total orders, or votes) as follows.
	\begin{center}
		5 : $a > c > b > e > d$,\\
		3 : $b > a > e > c > d$,\\
		2 : $c > d > b > a > e$.
	\end{center}
	
	We are asked to compute the winner according to some voting rules. 
	Plurality, veto and Borda are examples of commonly used voting rules.
	For each vote, plurality assigns score 1 to the top ranked alternative
	and 0 to the others, veto assigns score 0 to the bottom ranked alternative
	and 1 to the others, and Borda assigns score $m-i$ to the $i$th ranked 
	alternative ($m$ is the number of alternatives).
	Then the winner, thus, is the alternative with the highest score; in case of
	ties, we break them based on the order $a > b > c > d > e$.
	We compute the winner for $P_X$ as follows.
	
	\begin{enumerate}  \itemsep -4pt
		\item Plurality: alternative $a$ is the winner because the score of $a$ is maximal.
		\item Veto: since $a$, $b$ and $c$ are not vetoed but $d$ and $e$ are, 
					$a$, $b$ and $c$ are tied and $a$ wins.
		\item Borda: the winner is $a$ since its score, 31, is the highest.
	\end{enumerate}


While in the cases when the
number of alternatives is small the preference-aggregating problems (winner determination, 
etc) have received wide attention in the literature,
the problems concerning preferences over combinatorial
domains have not been as much investigated.
For example, a group of students plan on a joint vacation.
They may consider issues $Time$, $Transportation$,
$Destination$, $Duration$, etc.  
Each issue has a domain of values that it can take, e.g.,
$Time$ may have three values $spring$, $summer$ and $fall$.
There could be tens or hundreds of vacations even for
a relatively small number of issues, and
students will soon find it unlikely to enumerate all of
them in a strict total order.
Thus, applying voting rules directly is infeasible.
Methods that consider individual issues one by one have been proposed
and analyzed based on how well they approximate standard voting rules
\cite{fargier:ibi,Xia:SMV}.

Consequently, an expressive yet concise representation is needed in which
preferences over combinatorial alternatives are specified
in a compact way.
Such models have been proposed in the literature and include
conditional preference networks (CP-nets) \cite{Kaci:Pref}, 
lexicographic preference trees (LP trees) \cite{booth:learningLP} and
answer set optimization theories (ASO theories) \cite{Brewka:ASO,Brewka04}.
Compact and expressive as they are, CP-nets are limited
to describe preference relations under a 
\textit{ceteris paribus} interpretation, and
ASO theories apply a rather weak Pareto ordering
to rank outcomes with respect to multiple objectives \cite{Brewka:ASO}.

In our earlier work \cite{LiuT}\footnote{A joint paper with my PhD advisor
Dr. Miroslaw Truszczynski}, 
we have considered the preference paradigm of LP trees.
An LP tree is an intuitive and compact representation
of preferences over alternatives from combinatorial domains.
Informally, it exploits a natural way that humans apply to
express preference information in the setting of combinatorial domains.
Often a human agent would first consider the most important issue and
specify her preference within the domain of that issue.  Then,
based on how the most important issue is evaluated in an alternative,
the agent identifies the next important issue and specify her preference
over values of that issue, and so on.
This process ends up with a structured preference system that always
induces a strict total order.
In our paper, we obtained new complexity results for some preference
aggregation problems (e.g. computing the winning alternative)
according to positional scoring rules when votes are expressed as
LP trees, and provided insight into effectiveness of computational
tools on solving the problems of practical sizes.

For the future research, I will explore the relationship
between different preference formalisms, especially between
CP-nets and LP trees.
Furthermore, I will extend the existing
preference paradigms to allow more expressivity when it
comes to representing complex preferences.
In particular, I focus on a hybrid system where
relations of subsystems are specified as a decision tree
and preferences within each subsystem are expressed as
partial preorders.
We will also study preference learning techniques to
elicit preference relations over combinatorial domains,
aggregating partial LP trees, and manipulation problems
in social choice when votes are LP trees.

The outline of the rest of this proposal is the following. 
In \secref{relwork}, we considered related work that proposed
approaches to preference representation and aggregation both
in AI and social choice theory.
In \secref{mywork}, we focus on a specific preference formalism,
LP trees, and present our own work on aggregating LP trees
over combinatorial domains based on standard voting schemes, 
where both computational complexity
and expirical results are shown and analyzed. 
Plan of the following research towards my PhD thesis
is demonstrated in \secref{plan}.
