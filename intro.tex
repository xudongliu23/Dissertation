%%%% INTRO FUN! %%%%%%
%% start of comments
%\tc{
%	Here I will talk about the field I am working in, namely, 
%	preferences in Artificial Intelligence and 
%	Social Choice.
%	I will discribe the problems I am interested in in the field and explain
%	why these problems are interesting, important and
%	challenging for researchers to study. I will briefly mention other
%	people's work.
%	I will outline this document at the end of this section.
%}
%% end of comments
Preferences are ubiquitous.
They arise when we select ice-cream flavors, vote for candidates for an office,
and buy cars.
Preferences have spurred research in areas such as 
artificial intelligence, psychology, economics, and operations research.
Specifically, there has been growing interest in research problems on
preferences in contexts such as knowledge representation and reasoning, constraint 
satisfaction, decision making, and social choice theory. 
My research focuses on problems in preference modeling, 
reasoning and learning.

Preferences can be represented in a \tit{quantitative} and \tit{qualitative} manner.
For the former, agents express preferences in a numerical form of
a value function that precisely assesses the degree of satisfaction
of objects
(often called \textit{outcomes} or \textit{alternatives}).
Specifying preferences as value functions on alternatives
is feasible for humans in some situations, e.g., 
when the number of alternatives is limited.
In other circumstances, particularly when the number of alternatives is large,
people often cannot express their 
preferences directly and accurately as value 
functions \cite{Domshlak20111037}.
%due to the burden of deciding the value function for
%a large number of alternatives \cite{Domshlak20111037}.

Assume an agent is given three flavors of ice-cream:
$strawberry$, $chocolate$ and $vanilla$, and is asked to describe 
her preference among them.
The agent could think of a value function that
assigns quantities (\textit{utilities}) to each outcome
based on a scale from 1 to 10, with 10 representing the most satisfaction.
For instance, the agent could give the following value function:
\begin{center}
	$strawberry \mapsto 6$, $chocolate \mapsto 9$ and $vanilla \mapsto 3$.
\end{center}
This function shows that the favorite alternative to the agent is
$chocolate$ (it has the highest utility), and
$strawberry$ is preferred over $vanilla$.

Instead of rating alternatives quantitatively, it is
often easier and more intuitive to give preferential information
in a qualitative way,
so that to specify a binary preference relation.
Thus, the same agent could rank the flavors as the following
preference order:
\begin{center}
	$chocolate \succ strawberry \succ vanilla$.
\end{center}
Note that one can obtain the qualitative preferences from
the value function, but not vice versa.
My research deals with qualitative preference relations.

%Provided with preferences learned and presented in a formalism,
%what can we say about the agent's preferences over outcomes?
Preferences play an essential role in areas that involve making
decision, such as decision theory and social choice theory.
Once we have preferences of a user or users, we can reason about
the preferences to support decision making.
In general, preference reasoning problems can be classified
based on the number $n$ of agents from which the preferences are gathered:
\begin{enumerate} \itemsep -4pt
	\item $n=1$: individual decision making,
	\item $n>1$: collaborative decision making.
\end{enumerate}
In case $n=1$, we focus on optimization of the agent's preferences and
help her make a better decision by, for example, computing an optimal
alternative or comparing two given alternatives.  For the case
where $n>1$, it is important to calculate
a consensus (e.g., a winning outcome or ranking)
of the group of agents.

One of the problems in preference reasoning is to
aggregate preferences of a group of
agents. The problem is central to collective decision making and has been studied
extensively in social choice theory. 
Let us consider a scenario, where we are
given a set of alternatives $X = \{ a,b,c,d,e \}$ and a set $P_X$ of 10 
preferences (called \tit{votes}) as follows.
	\begin{center}
		5 : $a > c > b > e > d$,\\
		3 : $b > a > e > c > d$,\\
		2 : $c > d > b > a > e$.
	\end{center}
	
Note that the integer associated with every preference order
is the number of agents sharing that same preference.
We are asked to compute the winning alternative according to some aggregation rule. 
Plurality, veto and Borda are examples of commonly used voting rules.
For instance, Borda rule assigns score $m-i$ to the $i$th ranked 
alternative, where $m$ is the number of alternatives.
Thus, the winner is the alternative with the highest score.
We compute that the Borda winner for $P_X$ is $a$, since its 
score $31$ is the highest, followed by candidates $b$ and $c$
with the second highest score $26$.

While in the cases when the
number of alternatives is small the preference-aggregating 
problems, such as dominance testing and winner determination,
have received wide attention in the literature,
the problems concerning preferences over \tit{combinatorial
domains}, which typically contain large numbers of outcomes, 
have not been investigated as much.

To illustrate the setting of combinatorial domains, let us consider
a taxi company plans to purchase a fleet of cars.
The features (or, as we will say, attributes) that will be
taken into account are \tit{BodyType}, \tit{Capacity},
\tit{Make}, \tit{Price}, and \tit{Safety}.  
Each attribute has a domain of values that it can take, e.g.,
\tit{BodyType} may have four values \tit{minivan}, \tit{sedan}, \tit{sport}, and \tit{suv}.
There could be hundreds or thousands of cars,
described by different combinations of values on these five attributes, 
even for a relatively small number of attributes, and
the decision makers will soon find it impossible to enumerate all of
them from the most preferred to the least.

Consequently, an expressive yet concise representation is needed to specify
preferences over combinatorial alternatives.
Such preference formalisms are often categorized into \tit{logical models} and
\tit{graphical models}.
Logical models include penalty logic (\tit{Pen-logic}) \cite{haddawy1992representations}, 
possibilistic logic (\tit{Poss-logic}) \cite{DuboisLP91},
qualitative choice logic (\tit{Qual-logic}) \cite{brewka2004qualitative},
conditional preference theories (\tit{CP-theories}) \cite{Wilson04extendingcp-nets},
and answer set optimization (\tit{ASO}) \cite{Brewka:ASO},
whereas graphical models found in the literature include
generalized additive independence networks (GAI-nets) \cite{BacchusG95,LIP61766},
lexicographic preference trees (\tit{LP-trees}) \cite{booth:learningLP,conf/adt13/LiuT},
conditional preference networks (\tit{CP-nets}) \cite{Kaci:Pref},
conditional preference networks with trade-offs (\tit{TCP-nets}) \cite{BrafmanD02:TCP},
and conditional importance networks (\tit{CI-nets}) \cite{Kaci:Pref}.

Once we fix a preference formalism, say $\cF$, in which preferences of agents
are specified, eliciting and learning preference expressions in $\cF$ from agents
becomes a fundamental problem.
Different techniques have been proposed to preference learning
in $\cF$
such as \tit{active learning} (or \tit{preference elicitation}) and 
\tit{passive learning} \cite{Furnkranz:pref_learning}.
In the process of
active learning, the algorithm iteratively asks the user for a pairwise 
preference between two given outcomes and constructs
an instance of $\cF$ as more preferences are elicited.
For passive learning, the learning algorithm assumes that a set of
pairwise preferences are obtained over a period of time and builds an instance
of $\cF$ with no more information from the user.

My research has centered around the language of LP-trees.
Extending LP-trees, I have proposed two new tree-like preference formalisms:
partial lexicographic preference trees (\tit{PLP-trees}) \cite{conf/aaai15/LiuT}
and preference trees (\tit{P-trees}) \cite{fraser1994,liu2014preference,conf/adt15/liuT}.
The language of P-trees exploits a natural way humans apply to
express preference information in the setting of combinatorial domains.
Often a human agent would first consider the most desired \tit{criterion},
possibly represented by a propositional formula $\varphi$.
Outcomes that agree with it are preferred to those that do not.  
Then, the same mechanism is applied recursively to further discriminate 
among the outcomes that satisfy $\varphi$ and among those that falsify $\varphi$.
This process ends up with a structured preference system that always
induces a total preorder.

My research on tree-like preference formalisms can be categorized into
three main directions: preference modeling, preference learning and 
reasoning about preferences.

\smallskip \noindent \textbf{Preference Modeling \ }
\noindent My research formally proposed 
PLP-trees \cite{conf/aaai15/LiuT} and P-trees \cite{liu2014preference,conf/adt15/liuT}.
In particular, I studied
the relationship between P-trees and other existing preference languages, and
showed that P-trees extend
LP-trees, possibilistic logic, and ASO rules.
Moreover, my work established computational complexity results of commonly considered decision
problems in the setting of P-trees, such as \textit{dominance testing} 
\textit{optimality testing}, and
\textit{optimality testing w.r.t a property}.

\smallskip \noindent \textbf{Preference Learning \ }
Given a set of pairwise preferences between alternatives, 
called \textit{examples}, acquired from the user,
it is important to learn (i) a PLP-tree, 
preferably of a small size, 
consistent with a dataset of examples, and (ii) a PLP-tree correctly
ordering as many of the examples as possible in case
of inconsistency. In my work, I studied both these problems \cite{conf/aaai15/LiuT}. 
I established complexity results
for them and, in each case where the problem
is in the class P, proposed a polynomial time algorithm.
On the experimentation side,
I have designed and implemented algorithms, using both
Answer-Set Programming (ASP) and approximation methods, to learn PLP-trees 
and forests of these trees in the passive learning setting.
To facilitate experimentation, I developed several datasets
based on classification datasets such as \tit{Library for Preferences},
\tit{Preference Learning Site}, and \tit{UCI Machine Learning Repository}.
To evaluate the effectiveness and feasibility of our own models, I 
compared them with machine learning models, such as 
decision trees and random forests.

\smallskip \noindent \textbf{Preference Aggregation  \ }
In this area, I investigated two preference-aggregation 
problems, the \emph{winner} problem and the \emph{evaluation} problem,
based on \textit{positional scoring rules} (such as $k$-approval and Borda) 
when votes in elections are given as LP-trees \cite{abs/ijcai13dc/Liu,conf/adt13/LiuT}. 
My work brought new computational complexity results of these problems, and
provided computational methods to model and solve the problems using
\tit{answer set programming} (ASP) and \tit{weighted partial maximum satisfiability} (WPM).
%%%%%%%% FUTURE WORK
%I am also studying problems related to vulnerability of collective decisions under 
%misrepresentation of preferences specified over combinatorial domains.
%For instance, the \textit{coalitional manipulation problem}
%asks to decide if a small coalition set of manipulative
%agents can make some candidate a winner.
%I am examining positional scoring rules,
%comparison-based rules (e.g., the Copeland, Simpson and Maximin rules), and 
%distance-based rules (e.g., the Kemeny and Dodgson rules), for LP-trees,
%and will extend these results to elections over complicated domains to more general cases.
%%%%%%%% FUTURE WORK

%\smallskip \noindent \textbf{Application of Preferences  \ }
%Besides addressing computational problems on preferences,
%I have applied the theory into practical problem solving.
%One collaboration with research peers resulted in a refereed publication \cite{conf/adt13/Spradling},
%where we introduced a new variant of hedonic coalition formation games in which agents 
%have preferences on their own coalitions.
%In another collaborative effort with researchers in the Palo Alto Research Center,
%we designed and developed software modules for representing and reasoning about user 
%constraints and preferences in a trip planning setting, which is prepared
%for publication.

%%%%%%%% FUTURE WORK
%My long-term research goal is to study computational problems related to preferences, and 
%develop applications that help people or software agents make better decisions.
%Particularly, I intend to embed theories and practices on preferences into areas including
%data science, automated planning and scheduling, and database systems.
%%%%%%%% FUTURE WORK

The outline of the remainder of this thesis is the following. 
In \chref{preliminary}, I present necessary technical preliminaries
including binary relations, order theory, and computational complexity theory.
In \chref{relwork}, I go through related work that proposed
approaches to preference modeling and reasoning
in artificial intelligence and social choice theory.
In Chapters 4 to 7, I discuss results of my work on
modeling, learning and reasoning about preferences over combinatorial domains.
I conclude with a brief note in \chref{summary} on my ongoing research,
as well as on possible directions of future work.
