%%%% INTRO FUN! %%%%%%
%% start of comments
%\tc{
%	Here I will talk about the field I am working in, namely, 
%	preferences in Artificial Intelligence and 
%	Social Choice.
%	I will discribe the problems I am interested in in the field and explain
%	why these problems are interesting, important and
%	challenging for researchers to study. I will briefly mention other
%	people's work.
%	I will outline this document at the end of this section.
%}
%% end of comments
Preferences are ubiquitous in various forms (e.g., movie ratings, 
favorite ice cream flavors, and electoral ballots over candidates),
and attractive to researchers and scientists in areas such as 
artificial intelligence, psychology, economics, and operations research.
Specifically, there have been growing interests in research problems on
preferences in contexts such as knowledge representation and reasoning, constraint 
satisfaction, decision making, and social choice theory. 
My research focuses on problems in preference modeling, 
reasoning and learning.

Preferences can be represented in a quantitative or qualitative manner.
For the former, agents express preferences in a numerical form of
a value function that precisely assesses the degree of satisfaction
of objects
(often called \textit{outcomes} or \textit{alternatives}).
Specifying preferences as value functions on alternatives
is feasible for humans in some situations, e.g., 
when the number of alternatives is limited.
In other circumstances, particularly when the number of alternatives is large, however,
people often cannot express their 
preferences directly and accurately as value 
functions \cite{Domshlak20111037}.
%due to the burden of deciding the value function for
%a large number of alternatives \cite{Domshlak20111037}.

Assume an agent is given three flavors of ice-cream 
$strawberry$, $chocolate$ and $vanilla$, and asked to describe 
her preference among them.
The agent could think of a value function that
assigns quantities (\textit{utilities}) to each outcome
based on a scale from 1 to 10, with 10 representing the most satisfaction.
For instance, the agent could give the following value function:
\begin{center}
	$strawberry \mapsto 6$, $chocolate \mapsto 9$ and $vanilla \mapsto 3$.
\end{center}
This function shows that the favorite alternative to the agent is
$chocolate$ with the highest utility, and
$strawberry$ is preferred over $vanilla$.

Instead of rating alternatives quantitatively, it is
often easier and more intuitive to give preferential information
in small pieces in a qualitative way,
e.g., to specify binary preference relations.
Thus, the same agent could rank the flavors as the following
set of binary comparisons:
\begin{center}
	$\{chocolate \succ vanilla, chocolate \succ strawberry,
	strawberry \succ vanilla\}$.
\end{center}
Note that one can formulate the qualitative preferences from
the value function, but not vice versa.

My research mostly deals with preference models that induce
qualitative preference relations.
These preference formalisms are often categorized into \tit{logical models} and
\tit{graphical models}.
Logical models include penalty logic (\tit{Pen-logic})\cite{haddawy1992representations}, 
possibilistic logic (\tit{Poss-logic})\cite{DuboisLP91},
qualitative choice logic (\tit{Qual-logic})\cite{brewka2004qualitative},
conditional preference theories (\tit{CP-theories})\cite{Wilson04extendingcp-nets},
and answer set optimization theories (\tit{ASO-theories})\cite{Brewka:ASO},
whereas graphical models found in the literature are
generalized additive independence networks (GAI-nets)\cite{BacchusG95,LIP61766},
lexicographic preference trees (\tit{LP-trees})\cite{booth:learningLP,conf/adt13/LiuT},
%partial lexicographic preference trees (\tit{PLP-trees})\cite{conf/aaai15/LiuT},
%preference trees (\tit{P-trees})\cite{fraser1994,liu2014preference,conf/adt15/liuT},
conditional preference networks (\tit{CP-nets})\cite{Kaci:Pref},
conditional preference networks with tradeoffs (\tit{TCP-nets})\cite{BrafmanD02:TCP},
conditional importance networks (\tit{CI-nets})\cite{Kaci:Pref},
to list a few.
% Each of these formalisms provides the user
% with a concise way to express her preferences. 

Once we fix a preference formalism, say $\cF$, in which preferences of agents
are specified, acquiring preferences expressed in $\cF$ from agents
becomes a problem that has engaged attention from many AI researchers.
Different techniques have been proposed to preference learning
in $\cF$
such as \tit{active learning} (or\tit{ preference elicitation}) and 
\tit{passive learning}\cite{Furnkranz:pref_learning}.
In the process of
active learning, the algorithm iteratively asks the user for a pairwise 
preference between two given outcomes and constructs
an instance of $\cF$ as more preferences are elicited.
For passive learning, the learning algorithm assumes that a set of
pairwise preferences are obtained over a period of time and builds an instance
of $\cF$ with no more information from the user.

Provided with preferences learned and presented in a formalism,
what can we say about the agent's preferences over outcomes?
Reasoning problems can be classified
based on the number of agents, $n$, whose preference models we
are reasoning about at hand:
\begin{enumerate} \itemsep -4pt
	\item $n=1$: individual decision making,
	\item $n>1$: collaborative decision making.
\end{enumerate}
In case $n=1$, we focus on optimization of the agent's preferences and
help her make a better decision by, for example, computing an optimal
alternative or comparing two given alternatives.  For the case
where $n>1$, it is important to calculate
a collective decision (e.g., a winning outcome or ranking)
among the group of agents.

One of the problems in preference reasoning is to
aggregate preferences of a group of
agents (referred
to as \emph{voters} in social choice), which is central to collective decision making and has been studied
extensively in social choice theory. 
Let us consider such a scenario, where we are
given a set of alternatives $X = \{ a,b,c,d,e \}$ and a set $P_X$ of 10 
preferences (total orders, or votes) as follows.
	\begin{center}
		5 : $a > c > b > e > d$,\\
		3 : $b > a > e > c > d$,\\
		2 : $c > d > b > a > e$.
	\end{center}
	
	We are asked to compute the winner according to some voting rules. 
	Plurality, veto and Borda are examples of commonly used voting rules.
	For each vote, plurality assigns score 1 to the top ranked alternative
	and 0 to the others, veto assigns score 0 to the bottom ranked alternative
	and 1 to the others, and Borda assigns score $m-i$ to the $i$th ranked 
	alternative ($m$ is the number of alternatives).
	Then the winner, thus, is the alternative with the highest score; in case of
	ties, we break them based on the order $a > b > c > d > e$.
	We compute the winner for $P_X$ as follows.
	
	\begin{enumerate}  \itemsep -4pt
		\item Plurality: alternative $a$ is the winner because the score of $a$ is maximal.
		\item Veto: since $a$, $b$ and $c$ are not vetoed but $d$ and $e$ are, 
					$a$, $b$ and $c$ are tied and $a$ wins.
		\item Borda: the winner is $a$ since its score, 31, is the highest.
	\end{enumerate}


While in the cases when the
number of alternatives is small the preference-aggregating problems (winner determination, 
etc) have received wide attention in the literature,
the problems concerning preferences over combinatorial
domains have not been as much investigated.

For example, a committee board in a taxi company plan to purchase a fleet of
cars.
Each vehicle consists of attributes $BodyType$, $Capacity$,
$Make$, $Price$, $Safety$, etc.  
Each attribute has a domain of values that it can take, e.g.,
$BodyType$ may have four values \tit{minivan}, \tit{sedan}, \tit{sport}, and \tit{suv}.
There could be hundreds or thousands of cars even for
a relatively small number of attributes, and
the agent will soon find it impossible to enumerate all of
them from the most preferred to the least.
Thus, applying voting rules directly is infeasible.
Methods that consider individual issues one by one have been proposed
and analyzed based on how well they approximate standard voting rules
\cite{fargier:ibi,Xia:SMV}.

Consequently, an expressive yet concise representation is needed in which
preferences over combinatorial alternatives are specified
in a compact way.
Such models have been proposed in the literature and include
conditional preference networks (CP-nets) \cite{Kaci:Pref}, 
lexicographic preference trees (LP-trees) \cite{booth:learningLP} and
answer set optimization theories (ASO-theories) \cite{Brewka:ASO,Brewka04}.
Compact and expressive as they are, CP-nets are limited
to describe preference relations under a 
\textit{ceteris paribus} interpretation, and
ASO theories apply a rather weak Pareto ordering
to rank outcomes with respect to multiple objectives \cite{Brewka:ASO}.

Most of my research has considered the language of LP-trees.
Extending LP-trees, I have proposed two new tree-like preference formalisms:
partial lexicographic preference trees (\tit{PLP-trees})\cite{conf/aaai15/LiuT}
and preference trees (\tit{P-trees})\cite{fraser1994,liu2014preference,conf/adt15/liuT}.
For example, the language of P-trees exploits a natural way that humans apply to
express preference information in the setting of combinatorial domains.
Often a human agent would first consider the most desired \tit{criterion},
usually represented by a propositional formula $\varphi$.
Outcomes that agree with it are preferred to those that do not.  
Then, the same mechanism is applied recursively to further discriminate the 
outcomes that satisfy $\varphi$ and those that falsify $\varphi$.
This process ends up with a structured preference system that always
induces a total preorder.

My research has led to fruitful results as follows in the three general perspectives
of preferences, namely, preference modeling, learning and reasoning.

\smallskip \noindent \textbf{Preference Modeling \ }
\noindent My research formally defined the language 
of \tit{preference trees} (P-trees),
studied the relationship between P-trees and other existing preference languages, and
showed that P-trees extend  
LP-trees, possibilistic logic theories, and ASO-rules\cite{liu2014preference,conf/adt15/liuT}.
Moreover, my work established computational complexity results of commonly considered decision
problems in the setting of P-trees, such as \textit{dominance testing} 
\textit{optimality testing}, and
\textit{optimality testing w.r.t a property}.

\smallskip \noindent \textbf{Preference Learning \ }
My research introduced the formalism of \tit{partial lexicographic preference trees} (PLP-trees), 
a subclass of P-trees, over combinatorial domains\cite{conf/aaai15/LiuT}.
Given a set of pairwise preferences between alternatives, called \textit{examples}, from the user,
my work studied problems of how to learn (i) a PLP-tree, preferably of a small size, 
consistent with a dataset of examples, and (ii) a PLP-tree correctly
ordering as many of the examples as possible in case
of inconsistency. Furthermore, I established complexity results
of these problems and, in each case where the problem
is in the class P, proposed a polynomial time algorithm.
Continuing this research direction,
I am designing and implementing algorithms to learn these tree-based 
formalisms (e.g., LP-trees, PLP-trees, P-trees, and forests of these trees) 
for both passive learning and active learning.
To facilitate the learning process, we are developing datasets of examples
from various sources such as \tit{Library for Preferences},
\tit{Preference Learning Site}, and \tit{UCI Machine Learning Repository}.
To evaluate the effectiveness and feasibility of our own models, we plan to
compare and contrast them with machine learning models (e.g., decision trees, random forests).

\smallskip \noindent \textbf{Preference Aggregation  \ }
My research investigated two preference-aggregation 
problems, the \emph{winner} problem and the \emph{evaluation} problem,
based on \textit{positional scoring rules} (such as $k$-approval and Borda) 
when votes in elections are given as LP-trees\cite{abs/ijcai13dc/Liu,conf/adt13/LiuT}. 
My work presented new computational complexity results of these problems, and
provided computational methods to model and solve the problems using
\tit{answer set programming} (ASP) and \tit{weighted partial maximum satisfiability} (WPM).
I am also studying problems related to vulnerability of collective decisions under 
misrepresentation of preferences specified over combinatorial domains.
For instance, the \textit{coalitional manipulation problem}
asks to decide if a small coalition set of manipulative
voters can make some candidate a winner.
I am examining positional scoring rules,
comparison-based rules (e.g., the Copeland, Simpson and Maximin rules), and 
distance-based rules (e.g., the Kemeny and Dodgson rules), for LP-trees,
and will extend these results to elections over complicated domains to more general cases.

\smallskip \noindent \textbf{Application of Preferences  \ }
Besides addressing computational problems on preferences,
I have applied the theory into practical problem solving.
One collaboration with research peers resulted in a refereed publication\cite{conf/adt13/Spradling},
where we introduced a new variant of hedonic coalition formation games in which agents 
have preferences on their own coalitions.
In another collaborative effort with researchers in the Palo Alto Research Center,
we designed and developed software modules for representing and reasoning about user 
constraints and preferences in a trip planning setting, which is prepared
for publication.

My long-term research goal is to study computational problems related to preferences, and 
develop applications that help people or software agents make better decisions.
Particularly, I intend to embed theories and practices on preferences into areas including
data science, automated planning and scheduling, and database systems.

The outline of the rest of this proposal is the following. 
In \chref{preliminary}, I present necessary technical preliminaries
including binary relations, order theory, and computational complexity theory.
In \chref{relwork}, I will go through related work that proposed
approaches to preference modeling and reasoning
in artificial intelligence and social choice theory.
In \chref{PTrees},  \chref{learningPLPT}, \chref{aggLP}, and \chref{trip},
formal problems and results obtained in my research work are discussed
in modeling, learning and reasoning about preferences over combinatorial domains.
I will conclude with a brief note in \chref{summary} on my ongoing research,
as well as possible directions of future work.
