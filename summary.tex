%I now discuss my ongoing research work, as well as point out directions of
%future research.
%My ongoing research work that, once done, would be included in 
%the final version of my dissertation contains the following components:
%
%\begin{enumerate}  \itemsep -4pt
%	\item Preference learning and approximation:
%		\begin{itemize} \itemsep -5pt
%			\item Learning forests of lexicographic preference trees.
%			\item Approximating CP-nets using lexicographic preference trees.
%		\end{itemize}
%	\item Preference reasoning:
%		\begin{itemize} \itemsep -5pt
%			\item Aggregating partial lexicographic preference trees (PLP-trees).
%			\item Preference misrepresenting in elections where votes are LP-trees or PLP-trees.
%		\end{itemize}
%\end{enumerate}

The research in my thesis is about various aspects of preferences:
preference modeling, preference learning, and preference reasoning.
Preferences is a major research component studied
in artificial intelligence (AI) and decision theory, and is closely related to the 
social choice theory considered by economists and political scientists. 
In my thesis, I 
explore emerging connections between preferences in AI and social choice theory. 
Most of my research is on qualitative preference representations 
that extend and combine
existing formalisms such as  
lexicographic preference trees (LP-trees) \cite{booth:learningLP}, 
answer-set optimization theories (ASO-theories) \cite{Brewka:ASO}, 
possibilistic logic \cite{DuboisLP91}, and 
conditional preference networks (CP-nets) \cite{boutilier2004cp};
on learning problems that aim at discovering qualitative preference 
models and predictive preference information from practical data; and on
preference reasoning problems centered around qualitative preference optimization 
and aggregation methods.
Applications of my research include recommender systems, decision support tools,
multi-agent systems, and Internet trading and marketing platforms.

%Regarding future research directions, I will establish a research program of data-driven
%preference engineering, including data-driven preference learning, and preference
%reasoning and applications.
%
%For data-driven preference learning, I plan to study methods and algorithms below.
%\begin{enumerate}
%	\item Recommender Systems\cite{adomavicius2005toward}:
%		\begin{itemize}
%			\item Collaborative
%			\item Content-based
%			\item Hybrid
%		\end{itemize}
%	\item Machine Learning (fitting function):
%		\begin{itemize}
%			\item Supervised learning (e.g., decision trees, random forests)
%			\item Label ranking\cite{hullermeier2008label}
%		\end{itemize}
%	\item Model-based Learning (learning interpretable decision models):
%		\begin{itemize}
%			\item Preference Elicitation (Human-in-the-Loop)
%			\item Conditional Preference Networks, Preference Trees
%			\item Stochastic Models (e.g., Choquet integral\cite{tehrani2011choquistic}, 
%						TOPSIS-like models\cite{agarwal2014preference})
%		\end{itemize}
%\end{enumerate}
%
%For preference reasoning and applications, I propose to investigate both theoretic
%and practical problems in areas as follows.
%\begin{enumerate}
%	\item Social Choice and Welfare\cite{arrow2010handbook,Brandt:COMSOC}:
%		\begin{itemize}
%			\item Voting
%			\item Fair division
%			\item Strategy-proof Social Choice
%		\end{itemize}
%	\item Automated Planning and Scheduling\cite{son2006planning,bienvenu2011specifying,bast2015route}:
%		\begin{itemize}
%			\item Travel scheduling
%			\item Manufacturing
%			\item Traffic control
%		\end{itemize}
%\end{enumerate}
\section{Future Work}
My long-term research goal is to study computational problems related to preferences, and 
develop applications that help people or software agents make better decisions.
Particularly, I intend to embed theories and practices on preferences into areas including
data science, and automated planning and scheduling.

\smallskip \noindent \textbf{Data science \  } Discovering preference 
models from large data sets and reasoning about them 
can be of great value when decisions need to be customized for individual users.
For instance, e-Commerce companies want to make quality marketing decisions
on what customers would be interested in purchasing at a future time.
I propose to introduce contextual information and human-in-the-loop into existing learning methods
(e.g., collaborative filtering and content-based filtering used in recommender systems), 
in order to provide context-aware and user-centered predictions.
On the collective level, I mean to leverage social science methods (e.g., voting rules)
to combine individual models for joint decisions.
I plan to build preferential data sets and develop predictive systems, with collaborators
or sponsors from fields 
such as machine learning, computer vision, psychology, cognitive science, and behavioral science.

\smallskip \noindent \textbf{Automated planning and scheduling \  }
In planning and scheduling, constraints and preferences of agents may be more faceted than simply
``fastest" or ``cheapest."
I propose to design mathematical models allowing
intuitive representations of these individual accommodations. 
Furthermore, I intend to implement systems
that automate the acquisition of user constraints and preferences, 
and the computation of optimal plans or schedules 
based on these user-specific information.
This line of research potentially promotes collaboration with researchers of expertise in
travel scheduling, manufacturing, and traffic control.
