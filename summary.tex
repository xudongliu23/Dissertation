The research in my dissertation is about various aspects of preferences:
preference modeling, preference learning, and preference reasoning.
Preferences is a major research component studied
in artificial intelligence (AI) and decision theory, and is closely related to the 
social choice theory considered by economists and political scientists. 
In my dissertation, I 
explore emerging connections between preferences in AI and social choice theory. 
Most of my research is on qualitative preference representations 
that extend and combine
existing formalisms such as  
lexicographic preference trees (LP-trees) \cite{booth:learningLP}, 
answer-set optimization theories (ASO-theories) \cite{Brewka:ASO}, 
possibilistic logic \cite{DuboisLP91}, and 
conditional preference networks (CP-nets) \cite{boutilier2004cp};
on learning problems that aim at discovering qualitative preference 
models and predictive preference information from practical data; and on
preference reasoning problems centered around qualitative preference optimization 
and aggregation methods.
Applications of my research include recommender systems, decision support tools,
multi-agent systems, and Internet trading and marketing platforms.

I introduced partial lexicographic preference trees (PLP-trees) extending
the language of lexicographic preference trees (LP-trees).
I also proposed preference trees (P-trees) as a generalization
of PLP-trees.
Both PLP-trees and P-trees are intuitive qualitative preference languages
over combinatorial domains, and often compactly represent total preorders
over outcomes in such large domains.
I studied the expressive power of the two languages and showed that
they are closely related to existing preference formalisms.

For preference learning, my research focused on learning PLP-trees.
I studied various learning problems for PLP-trees and obtained
results on these problems both theoretically and experimentally.
My results showed that PLP-trees are highly accurate in modeling 
preferences arising in
practice, and can be effectively learned.
To reduce the overfitting of PLP-trees, I introduced the formalism
of PLP-forests, collections of PLP-trees.
My empirical results on learning PLP-forests showed that PLP-forests
are more expressive and accurate than PLP-trees.

Finally, for preference reasoning, I studied preference aggregation problems
(e.g., winner determination) in the setting of LP-trees, a special case
of PLP-trees that represent total orders.
Applying aggregation methods in social choice theory, I showed that
the aggregation problems are generally NP-hard.
For these hard problems, my empirical study using answer-set programming
(ASP) tools, designed specifically for solving NP-hard problems, showed
that ASP solvers are effective on large instances.

\section{Future Work}

My long-term research goal is to study computational problems related to preferences, and 
develop applications that help people or software agents make better decisions.
Particularly, I intend to embed theories and practices on preferences into areas including
data science, and automated planning and scheduling.

\smallskip \noindent \textbf{Data science \  } Discovering preference 
models from large data sets and reasoning about them 
can be of great value when decisions need to be customized for individual users.
For instance, e-Commerce companies want to make quality marketing decisions
on what customers would be interested in purchasing at a future time.
I propose to introduce contextual information and human-in-the-loop into existing learning methods
(e.g., collaborative filtering and content-based filtering used in recommender systems), 
in order to provide context-aware and user-centered predictions.
On the collective level, I mean to leverage social science methods (e.g., voting rules)
to combine individual models for joint decisions.
I plan to build preferential data sets and develop predictive systems, with collaborators
or sponsors from fields 
such as machine learning, computer vision, psychology, cognitive science, and behavioral science.

\smallskip \noindent \textbf{Automated planning and scheduling \  }
In planning and scheduling, constraints and preferences of agents may be more faceted than simply
``fastest" or ``cheapest."
I propose to design mathematical models allowing
intuitive representations of these individual accommodations. 
Furthermore, I intend to implement systems
that automate the acquisition of user constraints and preferences, 
and the computation of optimal plans or schedules 
based on these user-specific information.
This line of research potentially promotes collaboration with researchers of expertise in
travel scheduling, manufacturing, and traffic control.
