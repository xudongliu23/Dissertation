%\paragraph{\bf Abstract.}
%%%%We introduce \emph{preference trees} (\emph{P-trees}, for short) as 
%%%%a qualitative preference representation formalism. 
Preference trees, or \emph{P-trees} for short, offer an intuitive and
often concise way of representing preferences over combinatorial domains. 
In this chapter, we propose an alternative definition of P-trees, and 
formally define their compact representation that exploits occurrences 
of identical subtrees. We show that P-trees generalize lexicographic 
preference trees and are strictly more expressive. We relate P-trees 
to \tit{answer-set optimization} programs and \tit{possibilistic 
logic} theories. Finally, we study reasoning with P-trees and 
establish computational complexity results for key reasoning tasks
of comparing outcomes with respect to orders defined by P-trees, and 
of finding optimal outcomes.

\section{Introduction}
%Preferences are essential in areas such as constraint satisfaction, 
%decision making, multi-agent cooperation, Internet trading, and social
%choice. Consequently, preference representation languages and algorithms 
%for reasoning about preferences have received much 
%attention \cite{Kaci:Pref}. When there 
%are only a few objects (or \tit{outcomes}) to compare, it is both most 
%direct and feasible to represent preference orders by their explicit 
%enumerations. The situation changes when the domain of interest is 
%\emph{combinatorial}, that is, its elements are described in terms of 
%combinations of values of \emph{attributes}, say $x_1,\ldots, x_n$ (also 
%called \emph{variables} or \emph{attributes}), with each attribute $x_i$ 
%assuming values from some set $D_i$ --- its \emph{domain}. 
%
%Combinatorial domains appear commonly in applications. Since their 
%size is exponential in the number of attributes, they are often so large 
%as to make explicit representations of preference orders impractical. 
%Therefore, designing languages to represent preferences on elements 
%from combinatorial domains in a concise and intuitive fashion is 
%important. Several such languages have been proposed including penalty 
%and possibilistic logics \cite{DuboisLP91}, conditional preference
%networks (CP-nets) \cite{bbdh03}, lexicographic preference trees 
%(LP-trees) \cite{booth:learningLP}, and answer-set optimization 
%(ASO) programs \cite{Brewka:ASO}. 
%
%In this paper, we focus our study on combinatorial domains with binary 
%attributes. We assume that each attribute $x$ has the domain $\{x,\neg x\}$ 
%(we slightly abuse the notation here, overloading $x$ to 
%stand both for an attribute and for one of the elements of its domain). 
%Thus, outcomes in the combinatorial domain determined 
%by the set $\cI=\{x_1,\ldots,x_n\}$ of binary attributes are simply complete
%and consistent sets of literals over $\cI$. We denote the set of all 
%such sets of literals by $\CD(\cI)$. We typically view them as truth 
%assignments (interpretations) of the propositional language over the 
%vocabulary $\cI$. This allows us to use propositional formulas over 
%$\cI$ as concise representations of sets of outcomes over $\cI$. Namely, 
%each formula $\vph$ represents the set of outcomes that satisfy $\vph$ 
%(make $\vph$ true).

Let us consider preferences on the domain of cars. 
We will assume that cars are described by 
four binary variables: 
\begin{enumerate}  %\itemsep -2pt
\item \tit{Capacity} ($X_1$) with values \tit{high} ($1_1$) and 
\tit{low} ($0_1$),
\item \tit{Price} ($X_2$) with values \tit{high} ($1_2$) and 
\tit{low} ($0_2$),
\item \tit{Safety} ($X_3$) with values \tit{high} ($1_3$) and 
\tit{low} ($0_3$), and
\item \tit{Transmission} ($X_4$) could be \tit{automatic} ($1_4$)
and \tit{manual} ($0_4$).
\end{enumerate}
A truth assignment of these four variables $0_10_21_31_4$
represents the car with low capacity, low price,
high safety, and automatic transmission.

%To describe sets of vacations we can use formulas. For instance, 
%vacations that take place in the summer ($1_3$) or involve water sports
%($1_1$) can be described by the formula $X_3 \vee X_1$, and vacations 
%in Colorado ($0_2$) that we travel to by car ($1_4$) by the formula $\neg X_2
%\wedge X_4$.

Explicitly specifying strict preference orders on $\CD(\cI)$ becomes 
impractical even for combinatorial domains with as few as 7 or 8 attributes. 
However, 
the setting introduced above allows us to specify total preorders on
outcomes in terms of desirable properties outcomes should have. For 
instance, a formula $\vph$ might be interpreted as a definition of
a total preorder in which outcomes satisfying $\vph$ are preferred to
those that do not satisfy $\vph$ (and outcomes within each of these
two groups are equivalent). More generally, we could see an expression
(a sequence of formulas)
\[
\vph_1> \vph_2>\ldots>\vph_k
\]
as a definition of a total preorder in which outcomes satisfying $\vph_1$
are preferred to all others, among which outcomes satisfying $\vph_2$ are
preferred to all others, etc., and where outcomes not satisfying any of the 
formulas $\vph_i$ are least preferred. This way of specifying preferences 
is used
(with minor modifications) in possibilistic logic \cite{DuboisLP91} and 
ASO programs \cite{Brewka:ASO}.
In our example, the expression
\[
X_3 \land X_4 > \neg X_2 \land \neg X_4 
\]
states that we prefer automatic ($1_4$) cars with high safety ($1_3$)
to manual ($0_4$) cars with low price ($0_2$), with all 
other cars being the least preferred.

This linear specification of preferred formulas is sometimes too
restrictive. An agent might prefer outcomes that satisfy a property 
$\vph$ to those that do not. Within the first group that agent might 
prefer outcomes satisfying a property $\psi_1$ and within the other 
a property $\psi_2$. Such \emph{conditional} preference can be 
naturally captured by a form of a decision tree presented in 
\figref{pt}. Leaves, shown as boxes, represent sets of outcomes 
satisfying the corresponding conjunctions of formulas ($\vph\land\psi_1$,
$\vph\land\neg\psi_1$, etc.).

\begin{figure}[!ht]
	\centering
	  \begin{tikzpicture}[->,>=stealth',
	    level/.style={sibling distance=2cm/#1, level distance=25pt}]
	    \node [main node] (1){$\vph$}
	      child {node [main node,inner sep=1.7pt] (2) {$\psi_1$}
	      	child {node [rectangle,draw] (3) {}}
	      	child {node [rectangle,draw] (4) {}}
				}
	      child {node [main node,inner sep=1.7pt] (5) {$\psi_2$}
	      	child {node [rectangle,draw] (6) {}
					}
	      	child {node [rectangle,draw] (7) {}
					}
	      };
	  \end{tikzpicture}
  \caption{A preference tree}
%\vspace{-0.2cm}
  \label{fig:pt}
\end{figure}

%%%%We refer to trees such as the one in Figure \ref{fig:pt} as 
%%%%\emph{preference trees}, or \emph{P-trees}, for short. 
Trees such as the one in Figure \ref{fig:pt} are called \emph{preference 
trees}, or \emph{P-trees}. They were introduced by Fraser 
\cite{fraser1993,fraser1994}, who saw them as a convenient way to
represent conditional preferences. Despite their intuitive nature they
have not attracted much interest in the preference research in AI. In
particular, they were not studied for their relationship to other 
preference formalisms. The attribute of compact representations received 
only an informal treatment by Fraser (P-trees in their full representation
are often impractically large), and the algorithmic attributes of reasoning
with P-trees were also only touched upon. 


We propose an alternative definition of preference trees, 
and formally define their compact representation that exploits occurrences 
of identical subtrees. P-trees are reminiscent of LP-trees 
\cite{booth:learningLP}. We discuss the relation between the two concepts 
and show that P-trees offer a much more general, flexible and expressive 
way of representing preferences. We also discuss the relationship between 
preference trees and ASO preferences and possibilistic logic theories. 
We study the complexity of problems of comparing outcomes with respect 
to orders defined by preference trees, and of problems of finding optimal 
outcomes. 

This chapter is organized as follows. In the next section, we formally 
define P-trees and a compact way to represent them. In the following 
section we present results comparing the language of P-trees with other 
preference formalisms. We then move on to study the complexity of key 
reasoning tasks for preferences captured by P-trees and, finally, 
conclude by outlining some future research directions.


\section{Preference Trees}

In this section, we define preference trees and discuss their
representation. Let $\cI$ be a set of binary attributes\footnote{
	In case of multi-value attributes, $\cI$ is then a set of
	binary variables representing attribute values.
}. 
A \emph{preference tree} (\tit{P-tree}, for short) over $\cI$ is
a binary tree with all nodes other than 
leaves labeled with propositional formulas over $\cI$. Each P-tree 
$T$ defines a natural strict order $\succeq_T$ on the set of its leaves, 
the order of their enumeration from left to right. 

Given an outcome $o \in \CD(\cI)$, we define the \emph{leaf of $o$ in $T$}
as the leaf reached by starting at the root of $T$ and proceeding 
downwards. When at a node $t$ labeled with $\varphi$, if $o\models \varphi$, 
we descend to the left child of $t$; otherwise, we descend to the right node 
of $t$. We denote the leaf of $o$ in $T$ by $l_T(o)$.

We use the concept of the leaf of an outcome $o$ in a P-tree $T$ to 
define a total preorder on $\CD(\cI)$. Namely, for outcomes
$o_1, o_2\in \CD(\cI)$, 
we set $o_1\succeq_T o_2$, $o_1$ is \emph{preferred} to $o_2$, if $l_T(o_1) \succeq_T l_T(o_2)$, and
$o_1\succ_T o_2$, $o_1$ is \emph{strictly preferred} to $o_2$, 
if $l_T(o_1) \succ_T l_T(o_2)$.
(We overload the relations $\succeq_T$ and $\succ_T$ 
by using it both for the order on the leaves of $T$ 
and the corresponding preorder on the outcomes from $\CD(\cI)$). 
We say that $o_1$ is 
\emph{equivalent} to $o_2$, $o_1 \approx_T o_2$, if $l_T(o_1)=l_T(o_2)$. 
Finally, $o$ is \emph{optimal} if there exists no 
$o'$ such that $o' \succ_T o$.

Let us come back to the car example and assume that an agent 
prefers small cars with low price or big cars with high price
over the other options. This preference is described by the formula
$(X_1\land X_2) \lor (\neg X_1 \land \neg X_2)$ or, more concisely, as 
an equivalence $X_1 \equiv X_2$.
Within each of the two groups of cars 
(satisfying the formula and not satisfying the formula), 
high safety ($1_3$) is preferred. These preferences can be captured by the P-tree in
\figref{pt_vac_full}. We note that in this example, the preferences at 
the second level are \emph{unconditional}, that is, they do not depend on 
preferences at the top level. 

\begin{figure}[!ht]
        \centering
  \begin{subfigure}[b]{0.43\textwidth}
          \centering
          \begin{tikzpicture}[->,>=stealth',
            level/.style={sibling distance=1.7cm/#1, level distance=33pt}]
            \node [main node,inner sep=0pt] (1){$x_1 \!\! \equiv \!\! x_2$}
              child {node [main node,inner sep=5pt] (2) {$x_4$}
                child {node [rectangle,draw,fill] (3) {} edge from parent node[left] {\small{$l_1$}}}
                child {node [rectangle,draw,fill] (4) {} edge from parent node[right] {\small{$l_2$}}}
                                }
              child {node [main node,inner sep=5pt] (5) {$x_3$}
                child {node [rectangle,draw] (6) {}
                                        }
                child {node [rectangle,draw] (7) {}
                                        }
              };
          \end{tikzpicture}
          \caption{Full \label{fig:pt_vac_full}}
        \end{subfigure}%
  \begin{subfigure}[b]{0.43\textwidth}
          \centering
          \begin{tikzpicture}[->,>=stealth',
            level/.style={sibling distance=2cm/#1, level distance=33pt}]
            \node [main node,inner sep=0pt] (1){$x_1 \!\! \equiv \!\! x_2$}
              child {node [main node,inner sep=5pt] (2) {$x_3$}
                        child {node [rectangle] (4) {} edge from parent[draw=none]}
                                };
          \end{tikzpicture}
          \caption{Compact \label{fig:pt_vac_compact}}
        \end{subfigure}
  \caption{P-trees on cars}
%\vspace{-0.2cm}
  \label{fig:pt_vac}
\end{figure}

To compare two outcomes, $o_1=0_10_20_31_4$ 
and $o_2=1_11_21_30_4$,
we walk down the tree and find that $l_T(o_1)=l_1$ and $l_T(o_2)=l_2$. 
Thus, we have $o_1 \succ_T o_2$ since $l_1$ precedes $l_2$.


\ignore{
For a more complicated example, let us assume that an 
agent prefers vacations that take place in summer or involve hiking 
($\vph_1=\neg x_1 \lor x_3$) to all others, and this is the most 
desirable property to her. Among those vacations that satisfy $\vph_1$, 
the agent prefers hiking vacations in Colorado ($\vph_2 =\neg x_1 \land 
\neg x_2$) over the remaining ones in that group. Provided $\vph_1$ is 
satisfied, this is her second most important consideration. Her next 
concern for vacations satisfying 
$\vph_1$ is the mode of transportation. She prefers driving to flying 
for summer vacations and flying to driving, otherwise ($\vph_3= (x_3 
\rightarrow x_4) \land (\neg x_3 \rightarrow \neg x_4)$). Among vacations
that do not have the property $\vph_1$, that is, the vacations that are
in winter and involve water sports, the planner prefers to drive to 
Florida for her vacation ($\vph_2'=x_2 \land x_4$). The resulting
preference preorder on vacations can be represented as a P-tree $T$ 
shown in \figref{PTree_full}. This preorder has six clusters of
equivalent outcomes (vacation choices) represented by the six leaves,
with the decreasing preference for clusters of outcomes associated 
the leaves as we move from left to right. To compare two outcomes, $M
=\neg x_1x_2\neg x_3\neg x_4$ and $M'=x_1\neg x_2x_3\neg x_4$,
we walk down the trees and find that $l_T(M)=l_3$ and $l_T(M')=l_4$. 
Thus, $M \succ_T M'$ since $l_3$ precedes $l_4$.

\begin{figure}[!ht]
	\centering
  \begin{subfigure}[b]{0.23\textwidth}
		\centering
	  \begin{tikzpicture}[->,>=stealth',
	    level/.style={sibling distance=2cm/#1, level distance=27pt}]
	    \node [main node,inner sep=1.7pt] (1){$\varphi_1$}
	      child {node [main node,inner sep=1.7pt,label={[xshift=-0.5cm, yshift=-0.6cm]$t$}] (2) {$\varphi_2$}
	        child {node [main node,inner sep=1.7pt] (3) {$\varphi_3$}
	      		child {node [rectangle,draw] (4) {}}
	      		child {node [rectangle,draw] (5) {}}
					}
	        child {node [main node,inner sep=1.7pt] (6) {$\varphi_3$}
	      		child {node [rectangle,draw,fill] (7) {} edge from parent node[left] {\small{$l_3$}}}
	      		child {node [rectangle,draw,fill] (8) {} edge from parent node[right] {\small{$l_4$}}}
					}
				}
	      child {node [main node,inner sep=0.9pt] (9) {$\varphi_2'$}
	      	child {node [rectangle,draw] (10) {}
					}
	      	child {node [rectangle,draw] (11) {}
					}
	      };
	  \end{tikzpicture}
		\caption{Full representation \label{fig:PTree_full}}
	\end{subfigure}%
  \begin{subfigure}[b]{0.23\textwidth}
		\centering
	  \begin{tikzpicture}[->,>=stealth',
	    level/.style={sibling distance=2cm/#1, level distance=27pt}]
	    \node [main node,inner sep=1.7pt] (1){$\varphi_1$}
	      child {node [main node,inner sep=1.7pt,label={[xshift=-0.5cm, yshift=-0.6cm]$t$}] (2) {$\varphi_2$}
	        child {node [main node,inner sep=1.7pt] (3) {$\varphi_3$}
	      		child {node [rectangle] (4) {} edge from parent[draw=none]}
					}
				}
	      child {node [main node,inner sep=0.9pt] (5) {$\varphi_2'$}
	      	child {node [rectangle] (6) {} edge from parent[draw=none]}
	      };
	  \end{tikzpicture}
		\caption{Compact representation \label{fig:PTree_compact}}
	\end{subfigure}
  \caption{P-trees}
%\vspace{-0.2cm}
  \label{fig:PTree}
\end{figure}
}

The key property of P-trees is that they can represent any total preorder on 
$\CD(\cI)$.

\begin{prop}
\label{prop:1}
For every set $\cI$ of binary attributes, for every set $D\subseteq\CD(\cI)$
of outcomes over $\cI$, and for every total preorder $\succeq$ on $D$ into
no more than $2^n$ clusters of equivalent outcomes, there is a P-tree $T$ 
of depth at most $n$ such that the preorder determined by $T$ on $\CD(\cI)$ 
when restricted to $D$ coincides with $\succeq$ 
(that is, ${\succeq_T}_{|D}=\succeq$).
\end{prop}
\begin{proof}
Let $\succeq$ be a total preorder on a subset $D\subseteq \CD(\cI)$ of outcomes
over $\cI$, and let $D_1\succ D_2\succ \ldots\succ D_m$ be the corresponding 
strict ordering of clusters of equivalent outcomes, with $m\leq 2^n$. If
$m=1$, a single-leaf tree (no decision nodes, just a box node) represents 
this preorder. This tree has depth 0 and so, the assertion holds. Let us assume
then that $m>1$, and let us define $D'=D_1\cup\ldots\cup D_{\lceil m/2\rceil}$ 
and $D''=D\setminus
D'$. Let $\varphi_{D'}$ be a formula such that models of $D'$ 
are precisely the outcomes in $D'$ (such a formula can be constructed as a
disjunction of conjunctions of literals, each conjunction representing a
single outcome in $D'$). If we place $\varphi_{D'}$ in the root of a P-tree,
that tree represents the preorder with two clusters, $D'$ and $D''$, with 
$D'$ preceding $D''$. Since each of $D'$ and $D''$ has no more than $2^{n-1}$
clusters, by induction, the preorders $D_1\succ \ldots\succ 
D_{\lceil m/2\rceil}$ and $D_{\lceil m/2\rceil+1}\succ \ldots\succ D_m$ 
can each be represented as a P-tree with depth at most $n-1$. Placing
these trees as the left and the right subtrees of $\varphi_{D'}$ 
respectively results in a P-tree of depth at most $n$ that represents
$\succeq$. 
\end{proof}

\paragraph{\bf Compact Representation of P-Trees.}
Proposition \ref{prop:1} shows high expressivity 
of P-trees. However, the construction described in the proof has little 
practical use. First, the P-tree it produces may have a 
large size due to the large sizes of labeling formulas that are generated. 
Second, to apply it, one would need to have an explicit enumeration of 
the preorder to be modeled, and that explicit representation in practical 
settings is unavailable.         

However, preferences over combinatorial domains that arise in practice
typically have structure that can be elicited from a user and exploited
when constructing a P-tree representation of the preferences. First,
decisions at each level are often based on considerations involving
only very few attributes, often just one or two and very rarely more than that.
Moreover, the subtrees of a node that order the ``left'' and the``right''
outcomes are often identical or similar. 

Exploiting these features often leads to much smaller representations.  
A \emph{compact P-tree over} $\cI$ is a tree such that
\begin{enumerate}[itemsep=0pt]
	\item every node is labeled with a Boolean formula over $\cI$, and
  \item every non-leaf node $t$ labeled with $\varphi$ has either
        two outgoing edges, with the left one meant to be taken by 
        outcomes that satisfy $\varphi$ and the right one by those that
        make $\varphi$ false (\figref{compact1}), or one 
    outgoing edge pointing
    \begin{itemize}[itemsep=0pt]
      \item straight-down (\figref{compact2}), which indicates that the two subtrees of $t$ 
            are \tit{identical} and the formulas
            labeling every pair of corresponding nodes in the two subtrees are the \tit{same},
      \item left (\figref{compact3}), which indicates that right subtree 
      of $t$ is empty, or
      \item right (\figref{compact4}), which indicates that left subtree 
      of $t$ is empty.
    \end{itemize}
\end{enumerate}

\begin{figure}[!ht]
	\centering
  \begin{subfigure}[b]{0.3\textwidth}
		\centering
	  \begin{tikzpicture}[->,>=stealth',
	    level/.style={sibling distance=1.5cm/#1, level distance=30pt}]
	    \node [main node,inner sep=3pt,label={[xshift=0cm, yshift=0cm]$t$}] (1){$\varphi$}
				child [edge from parent path ={(\tikzparentnode.-140) -- (\tikzchildnode.north)}] {
					node [subtree,yshift=0.4cm] (2) {}}
				child [edge from parent path ={(\tikzparentnode.-40) -- (\tikzchildnode.north)}] {
					node [subtree,yshift=0.4cm] (3) {}};
	  \end{tikzpicture}
		\caption{\label{fig:compact1}}
	\end{subfigure}%
  \begin{subfigure}[b]{0.2\textwidth}
		\centering
	  \begin{tikzpicture}[->,>=stealth',
	    level/.style={sibling distance=1.5cm/#1, level distance=30pt}]
		    \node [main node,inner sep=3pt,label={[xshift=0cm, yshift=0cm]$t$}] (1){$\varphi$}
					child {node [subtree,yshift=0.4cm] (2) {}};
	  \end{tikzpicture}
		\caption{\label{fig:compact2}}
	\end{subfigure}%
  \begin{subfigure}[b]{0.2\textwidth}
		\centering
		  \begin{tikzpicture}[->,>=stealth',
		    level/.style={sibling distance=1.5cm/#1, level distance=30pt}]
		    \node [main node,inner sep=3pt,label={[xshift=0cm, yshift=0cm]$t$}] (1){$\varphi$}
					child [edge from parent path ={(\tikzparentnode.-140) -- (\tikzchildnode.north)}] {
						node [subtree,yshift=0.4cm] (2) {}}
	      		child {node [rectangle] (3) {} edge from parent[draw=none]
		      };
		  \end{tikzpicture}
			\caption{\label{fig:compact3}}
	\end{subfigure}%
  \begin{subfigure}[b]{0.2\textwidth}
		\centering
	  \begin{tikzpicture}[->,>=stealth',
	    level/.style={sibling distance=1.5cm/#1, level distance=30pt}]
		    \node [main node,inner sep=3pt,label={[xshift=0cm, yshift=0cm]$t$}] (1){$\varphi$}
	      	child {node [rectangle] (2) {} edge from parent[draw=none]
		      }
					child [edge from parent path ={(\tikzparentnode.-40) -- (\tikzchildnode.north)}] {
						node [subtree,yshift=0.4cm] (2) {}};
	  \end{tikzpicture}
		\caption{\label{fig:compact4}}
	\end{subfigure}
  \caption{Compact P-trees}
%\vspace{-0.2cm}
  \label{fig:compact}
\end{figure}


The P-tree in \figref{pt_vac_full} can be collapsed as both 
subtrees of the root are the same (including the labeling formulas). This 
leads to a tree in \figref{pt_vac_compact} with a straight-down edge.
\ignore{
Let $t$ be a node in a P-tree $T$.  
We denote by $\Inst(t)$ the set of 
ancestor nodes of $t$ in $T$ that have two outgoing edges.
}
\ignore{
Similarly, in the P-tree 
in \figref{PTree_full}, the two subtrees of node $t$ and the formulas 
labeling the corresponding nodes are identical. Thus, we can collapse 
them and achieve a compact representation in \figref{PTree_compact}. 
}
We note that we drop box-labeled leaves in compact representations 
of P-trees, as they no longer have an interpretation as distinct clusters.

\paragraph{\bf Empty Leaves in P-Trees.} Given a P-tree $T$ one can prune it
so that all sets of outcomes corresponding to its leaves are non-empty.
However, keeping empty clusters may lead to compact representations of much
smaller (in general, even exponentially smaller) size.

%A compact P-tree $T$ in \figref{PTree_EL_compact} represents a full 
%binary tree $T'$ in \figref{PTree_EL_full}.
%The formulas labeling the non-leaf nodes in $T$ are $\varphi_1=\neg x_1 \vee x_3$, $\varphi_2=x_2 \vee \neg x_4$
%and $\varphi_3=x_2 \wedge x_3$.
%We can check that leaves $l_1$, $l_2$ and $l_3$ are empty, that is,
%the conjunctions $\varphi_1 \wedge \neg \varphi_2 \wedge \varphi_3$,
%$\neg \varphi_1 \wedge \varphi_2 \wedge \varphi_3$ and
%$\neg \varphi_1 \wedge \neg \varphi_2 \wedge \varphi_3$ are
%unsatisfiable.
A full P-tree $T$ in \figref{PTree_EL_full} uses labels
$\varphi_1=\neg X_1 \vee X_3$, $\varphi_2=X_2 \vee \neg X_4$, and
$\varphi_3=X_2 \wedge X_3$.
We check that leaves $l_1$, $l_2$ and $l_3$ are empty, that is,
the conjunctions $\varphi_1 \wedge \neg \varphi_2 \wedge \varphi_3$,
$\neg \varphi_1 \wedge \varphi_2 \wedge \varphi_3$ and
$\neg \varphi_1 \wedge \neg \varphi_2 \wedge \varphi_3$ are
unsatisfiable.
Pruning $T$ one obtains a compact tree $T'$ (\figref{PTree_NEL_compact}) that is
smaller compared to $T$, but larger than $T''$ (\figref{PTree_EL_compact}), another compact 
representation of $T$, should we allow empty leaves and exploit
the structure of $T$.

\begin{figure}[!ht]
	\centering
  \begin{subfigure}[b]{0.4\textwidth}
		\centering
	  \begin{tikzpicture}[->,>=stealth',
	    level/.style={sibling distance=2cm/#1, level distance=27pt}]
	    \node [main node,inner sep=1.7pt] (1){$\varphi_1$}
	      child {node [main node,inner sep=1.7pt] (2) {$\varphi_2$}
	        child {node [main node,inner sep=1.7pt] (3) {$\varphi_3$}
	      		child {node [rectangle,draw] (4) {}}
	      		child {node [rectangle,draw] (5) {}}
					}
	        child {node [main node,inner sep=1.7pt] (6) {$\varphi_3$}
	      		child {node [rectangle,draw,fill] (7) {} edge from parent node[left] {\small{$l_1$}}}
	      		child {node [rectangle,draw] (8) {}}
					}
				}
	      child {node [main node,inner sep=1.7pt] (9) {$\varphi_2$}
	        child {node [main node,inner sep=1.7pt] (10) {$\varphi_3$}
	      		child {node [rectangle,draw,fill] (11) {} edge from parent node[left] {\small{$l_2$}}}
	      		child {node [rectangle,draw] (12) {}}
					}
	        child {node [main node,inner sep=1.7pt] (13) {$\varphi_3$}
	      		child {node [rectangle,draw,fill] (14) {} edge from parent node[left] {\small{$l_3$}}}
	      		child {node [rectangle,draw] (15) {}}
					}
	      };
	  \end{tikzpicture}
		\caption{$T$ \label{fig:PTree_EL_full}}
	\end{subfigure}%
  \begin{subfigure}[b]{0.4\textwidth}
		\centering
		  \begin{tikzpicture}[->,>=stealth',
		    level/.style={sibling distance=1.6cm/#1, level distance=27pt}]
		    \node [main node,inner sep=1.7pt] (1){$\varphi_1$}
		      child {node [main node,inner sep=1.7pt] (2) {$\varphi_2$}
		        child {node [main node,inner sep=1.7pt] (3) {$\varphi_3$}
		      		child {node [rectangle] (4) {} edge from parent[draw=none]}
		      		child {node [rectangle] (5) {} edge from parent[draw=none]}
						}
		        child {node [rectangle] (6) {} edge from parent[draw=none]
						}
					}
		      child {node [main node,inner sep=1.7pt] (7) {$\varphi_2$}
		        child {node [rectangle] (8) {} edge from parent[draw=none]}
		        child {node [rectangle] (9) {} edge from parent[draw=none]}
		      };
		  \end{tikzpicture}
			\caption{$T'$: pruned $T$\label{fig:PTree_NEL_compact}}
	\end{subfigure}%
  \begin{subfigure}[b]{0.2\textwidth}
		\centering
	  \begin{tikzpicture}[->,>=stealth',
	    level/.style={sibling distance=2cm/#1, level distance=27pt}]
	    \node [main node,inner sep=1.7pt] (1){$\varphi_1$}
	      child {node [main node,inner sep=1.7pt] (2) {$\varphi_2$}
	        child {node [main node,inner sep=1.7pt] (3) {$\varphi_3$}
	      		child {node [rectangle] (4) {} edge from parent[draw=none]}
					}
				};
	  \end{tikzpicture}
		\caption{$T''$ \label{fig:PTree_EL_compact}}

	\end{subfigure}
  \caption{P-trees with empty leaves}
%\vspace{-0.2cm}
  \label{fig:PTree_EL}
\end{figure}

That example generalizes and leads to the question
of finding small sized representations of P-trees.
(We conjecture that the
problem in its decision version asking about the existence of a compact 
representation of size at most $k$ is NP-complete). 
From now on, we assume that P-trees are given in their compact representation. 

%\begin{figure}[!ht]
%	\centering
%	  \begin{tikzpicture}[->,>=stealth',
%	    level/.style={sibling distance=2cm/#1, level distance=27pt}]
%	    \node [main node,inner sep=1.7pt] (1){$\varphi_1$}
%	      child {node [main node,inner sep=1.7pt] (2) {$\varphi_2$}
%	        child {node [main node,inner sep=1.7pt] (3) {$\varphi_3$}
%	      		child {node [rectangle,draw] (4) {}}
%	      		child {node [rectangle,draw] (5) {}}
%					}
%	        child {node [rectangle,draw] (6) {}
%					}
%				}
%	      child {node [main node,inner sep=1.7pt] (7) {$\varphi_2$}
%	        child {node [rectangle,draw] (8) {}}
%	        child {node [rectangle,draw] (9) {}}
%	      };
%	  \end{tikzpicture}
%		\caption{$T''$: pruned $T'$}
%\vspace{-0.2cm}
%  \label{fig:PTree_full_NEL}
%\end{figure}

\section{P-Trees and Other Formalisms}

In this section we compare the preference representation language of P-trees
with other preference languages.


\paragraph{\bf P-Trees Generalize LP-Trees.}

As stated earlier, P-trees are reminiscent of LP-trees, a preference 
language that has received significant attention recently 
\cite{booth:learningLP,lang:aggLP,LiuT}. In fact, LP-trees over a set
$\cI=\{X_1,\ldots,X_n\}$ of attributes are simply special P-trees over $\cI$.
Namely, an LP-tree over $\cI$ can be defined as a P-tree over $\cI$, in
which all formulas labeling nodes are atoms $x_i$ or their negations $\neg x_i$,
depending on whether $x_i$ or $\neg x_i$ is the preferred over the other, and every path from 
the root to a leaf has all atoms $x_i$ appear in it as labels exactly 
once. Clearly, LP-trees are full binary trees of depth $n$ (assuming the depth of the root is 1) 
and determine strict \emph{total orders} on outcomes 
in $\CD(\cI)$ (no indifference between different outcomes). An example 
of an LP-tree over $\{X_1,X_2,X_3,X_4\}$ for our car example is 
given in \figref{LPT_full}. 

\begin{figure}[!ht]
	\centering
	  \begin{tikzpicture}[->,>=stealth',
       level 1/.style={sibling distance=5.1cm, level distance=30pt},
       level 2/.style={sibling distance=2.5cm, level distance=30pt},
       level 3/.style={sibling distance=1.3cm, level distance=30pt}]
	    \node [main node,inner sep=2pt,label={[xshift=0.9cm, yshift=-0.7cm]$\scriptstyle \neg x_1>x_1$}] (1){$X_1$}
	    child {node [main node,inner sep=2pt,label={[xshift=0.9cm, yshift=-0.7cm]$\scriptstyle x_3>\neg x_3$}] (2) {$X_3$}
	      child {node [main node,inner sep=2pt,label={[xshift=0.9cm, yshift=-0.7cm]$\scriptstyle \neg x_2>x_2$}] (3) {$X_2$}
	        child {node [main node,inner sep=2pt,label={[xshift=-0.1cm, yshift=-1.2cm]$\scriptstyle \neg x_4>x_4$}] (4) {$X_4$}}
	        child {node [main node,inner sep=2pt,label={[xshift=0.05cm, yshift=-1.2cm]$\scriptstyle x_4>\neg x_4$}] (5) {$X_4$}}
				}
	      child {node [main node,inner sep=2pt,label={[xshift=0.9cm, yshift=-0.7cm]$\scriptstyle \neg x_4>x_4$}] (6) {$X_4$}
	        child {node [main node,inner sep=2pt,label={[xshift=0cm, yshift=-1.2cm]$\scriptstyle x_2>\neg x_2$}] (7) {$X_2$}}
	        child {node [main node,inner sep=2pt,label={[xshift=0.05cm, yshift=-1.2cm]$\scriptstyle x_2>\neg x_2$}] (8) {$X_2$}}
	      }
			}
	    child {node [main node,inner sep=2pt,label={[xshift=0.9cm, yshift=-0.7cm]$\scriptstyle x_3>\neg x_3$}] (9) {$X_3$}
	      child {node [main node,inner sep=2pt,label={[xshift=0.9cm, yshift=-0.7cm]$\scriptstyle \neg x_2>x_2$}] (10) {$X_2$}
	        child {node [main node,inner sep=2pt,label={[xshift=0cm, yshift=-1.2cm]$\scriptstyle \neg x_4>x_4$}] (11) {$X_4$}}
	        child {node [main node,inner sep=2pt,label={[xshift=0.05cm, yshift=-1.2cm]$\scriptstyle x_4> \neg x_4$}] (12) {$X_4$}}
				}
	      child {node [main node,inner sep=2pt,label={[xshift=0.9cm, yshift=-0.7cm]$\scriptstyle \neg x_4>x_4$}] (13) {$X_4$}
	        child {node [main node,inner sep=2pt,label={[xshift=0cm, yshift=-1.2cm]$\scriptstyle x_2>\neg x_2$}] (14) {$X_2$}}
	        child {node [main node,inner sep=2pt,label={[xshift=0.05cm, yshift=-1.2cm]$\scriptstyle x_2>\neg x_2$}] (15) {$X_2$}}
	      }
			}
			;
	  \end{tikzpicture}
  \caption{A full LP-tree on cars}
%	\vspace{-0.2cm}
  \label{fig:LPT_full}
\end{figure}

In general representing preferences by LP-trees is impractical. The size of
the representation is of the same order as that of an explicit enumeration
of the preference order. However, in many cases preferences on outcomes 
have structure that leads to LP-trees with similar subtrees. That structure
can
be exploited, as in P-trees, to represent LP-trees compactly.  
\figref{LPT_PT_lpt} shows a compact representation of the LP-tree in  
\figref{LPT_full}. We note the presence of conditional preference tables 
that make up for the lost full binary tree structure. Together with the 
simplicity of the language, compact representations are behind the practical 
usefulness of LP-trees. The compact representations of LP-trees translate 
into compact representations of P-trees, in the sense defined above. This 
matter is not central to our discussion and we simply illustrate it with 
an example. The compactly represented P-tree in \figref{LPT_PT_pt} is the
counterpart to the compact LP-tree in \figref{LPT_PT_lpt},
where $\varphi=(X_2 \land X_4) \lor (\neg X_2 \land \neg X_4)$.

\begin{figure}[!ht]
	\centering
  \begin{subfigure}[b]{0.43\textwidth}
		\centering
	  \begin{tikzpicture}[->,>=stealth',node distance=1.2cm]
	        
	    \node[main node,inner sep=2pt] (1) {$X_1$};
	    \node[rectangle,draw,inner sep=1pt] at (1.2,0) {$\neg x_1\!\! >\!\! x_1$};
	
	    \node[main node,inner sep=2pt] (2) [below of=1] {$X_3$};
	    \node[rectangle,draw,inner sep=1pt] at (1.2,-1.2) {$x_3 \!\!>\!\! \neg x_3$};
	    
	    \node[main node,inner sep=2pt] (3) [below left of=2] {$X_2$};
	    \node[rectangle,draw,inner sep=1pt] at (-2,-2.05) {$\neg x_2 \!\!>\!\! x_2$};
	    
	    \node[main node,inner sep=2pt] (4) [below of=3] {$X_4$};
	    \node[rectangle split, rectangle split parts=2, draw,inner sep=1pt,font=\sffamily\small] at (-2.4,-3.3)
	        {
	          $x_2\!:\!x_4 \!\!>\!\! \neg x_4$
	          \nodepart{second}
	          $\neg x_2\!:\!\neg x_4 \!\!>\!\! x_4$
	        };
	    
	    \node[main node,inner sep=2pt] (5) [below right of=2] {$X_4$};
	    \node[rectangle,draw,inner sep=1pt] at (2.05,-2.05) {$\neg x_4\!\! > \!\! x_4$};
	    
	    \node[main node,inner sep=2pt] (6) [below of=5] {$X_2$};
	    \node[rectangle,draw,inner sep=1pt] at (2.05,-3.3) {$x_2 \!\!>\!\! \neg x_2$};
	  
	    \path[every node/.style={font=\sffamily\small}]
	      (1) edge (2)
	      (2) edge (3)
	          edge (5)
	      (3) edge (4)
	      (5) edge (6);
	  \end{tikzpicture}
  	\caption{A compact LP-tree \label{fig:LPT_PT_lpt}}
	\end{subfigure}%
	\begin{subfigure}[b]{0.43\textwidth}
			\centering
		  \begin{tikzpicture}[->,>=stealth',
		    level/.style={sibling distance=3.4cm/#1, level distance=31pt}]
		    \node [main node,inner sep=0pt] (1){$\neg X_1$}
		    child {node [main node,inner sep=2.5pt] (2) {$X_3$}
		      child {node [main node,inner sep=0pt] (3) {$\neg X_2$}
		        child {node [main node,inner sep=5pt] (4) {$\varphi$}
		      		%child {node [rectangle] (5) {} edge from parent[draw=none]}
						}
					}
		      child {node [main node,inner sep=0pt] (6) {$\neg X_4$}
		        child {node [main node,inner sep=2.5pt] (7) {$X_2$}
		      		%child {node [rectangle] (8) {} edge from parent[draw=none]}
						}
		      }};
		  \end{tikzpicture}
  		\caption{The corresponding P-tree \label{fig:LPT_PT_pt}}
		\end{subfigure}
  \caption{A compact LP-tree as a compact P-tree}
%\vspace{-0.2cm}
  \label{fig:LPT_PT}
\end{figure}

The major drawback of LP-trees is that they can capture only a very small
fraction of preference orders. 
Let us first compute the number, say $G(n)$,
of LP-trees over $n$ attributes.
We have
\[
 G(n) =
  \begin{cases}
   1, & n=0; \\
   2n \cdot G^2(n-1), & n>0.
  \end{cases}
\]
From this recursive definition of $G(n)$, we calculate that
\begin{equation*}
	G(n)=\prod_{k=0}^{n-1} (n-k)^{2^k} \cdot 2^{2^k}
\end{equation*}
It is asymptotically much smaller than $L(n)=(2^n)!$, the number of all 
preference orders of the corresponding domain of outcomes. 
In particular, we show in \thmref{exp_small_ratio} that
LP-trees only encode an exponentially small portion of all linear orders.
\begin{thm}
\label{thm:exp_small_ratio}
	Let $L(n)=2^n!$ be the number of linear orders of outcomes over $n$ binary attributes,
	$r$ be the ratio of $G(n)$ to $L(n)$.
	We have
	\begin{equation}
		r = \frac{G(n)}{L(n)} < \frac{1}{2^{(2^n \cdot (n-\log n -2))}}.
	\end{equation}
\end{thm}
\begin{proof}
	\begin{align*}
		r 2^n! &= T(n); \numberthis \\
		\log r + \log 2^n! &= \log (\prod_{k=0}^{n-1} (n-k)^{2^k} \cdot 2^{2^k})\\
		&= \sum_{k=0}^{n-1} (\log ((n-k)^{2^k}) + 2^k)\\
		&= \sum_{k=0}^{n-1} (2^k \cdot (\log (n-k) + 1))\\
		&< \sum_{k=0}^{n-1} (2^k \cdot (\log n + 1))\\
		&= (\log n + 1) \cdot \sum_{k=0}^{n-1} 2^k\\
		&= (\log n + 1) \cdot (2^n-1). \numberthis 
	\end{align*}
	
	Let $N$ be such that $N=(\log n + 1) \cdot (2^n-1)$.
	By the Stirling's approximation $n! \geq \sqrt{2\pi n} \cdot (\frac{n}{e})^n$,
	we have the following.
	
	\begin{align*}
		\log r &< N - \log 2^n! \\
		&\leq N- \log (\sqrt{2\pi 2^n} \cdot (\frac{2^n}{e})^{2^n})\\
		&< N- \log (\sqrt{2^n} \cdot (\frac{2^n}{e})^{2^n})\\
		&= N- (\frac{n}{2}+\log \frac{2^{n\cdot 2^n}}{e^{2^n}})\\
		&= N- (\frac{n}{2}+n\cdot 2^n-\log e^{2^n})\\
		&< N- (\frac{n}{2}+n\cdot 2^n-\log 2^{2^n})\\
		&= N- (\frac{n}{2}+n\cdot 2^n-2^n)\\
		&= 2^n \cdot (\log n -n+2)-\log n-1-\frac{n}{2}\\
		&< 2^n \cdot (\log n -n+2). \numberthis
	\end{align*}
	
	Therefore, we have
	\begin{equation}
		r < \frac{1}{2^{(2^n \cdot (n-\log n -2))}}.
	\end{equation}
\end{proof}

This is in stark contrast with Proposition \ref{prop:1}, according
to which every total preorder can be represented by a P-tree.

Even very natural orderings, which have simple (and compact) representations
by P-trees often cannot be represented as LP-trees. For instance, there is 
no LP-tree on $\{x_1,x_2\}$ representing the order $00\succ 11 \succ 01 
\succ 10\}$. However, the P-trees (both full and compact) in \figref{pt_vac}
do specify it.

\paragraph{\bf P-Trees Extend ASO-Rules.} The formalism of ASO-rules 
\cite{Brewka:ASO} provides an intuitive way to express preferences
over outcomes as total preorders.
An ASO-rule partitions outcomes into ordered clusters according to 
the semantics of the formalism.
Formally, an ASO-rule $r$ over $\cI$ is a preference rule of the form
\begin{equation} \label{eq:ASO}
\small{C_1 > \ldots > C_m \leftarrow B,}
\end{equation}

\noindent
where all $C_i$'s and $B$ are propositional formulas over $\cI$.
%For each outcome $M$, rule (\ref{eq:ASO}) determines its \emph{satisfaction degree}.
%It is denoted by $\SD_r(M)$ and defined by
%\[
%\small{\SD_r(M) =
%  \begin{cases}
%   1, & M \models \neg B \\
%   m+1, & M \models B \wedge \bigwedge_{1 \leq i \leq m} \neg C_i \\
%   min\{i:M \models C_i\}, & \mbox{otherwise}.
%  \end{cases}}
%\]
%
%\noindent
%We say that an outcome $M$ is weakly preferred to an 
%outcome $M'$ ($M
%\succeq_r M'$) if $\SD_r(M)\leq\SD_r(M')$. Thus, the notion of the 
%satisfaction degree (or, equivalently, the preference $r$) partitions
%outcomes into (in general) $m+1$ clusters.\footnote{This definition is 
%a slight adaptation of the original one.}

Let us consider the domain of cars.
An agent may prefer a car with high safety and capacity to
one with low safety and capacity, if she
is going for an expensive car.
Such preference can be described as an ASO-rule:
\begin{center}
	$X_1 \land X_3 > \neg X_1 \land \neg X_3 \leftarrow X_2$.
\end{center}
Under the semantics of ASO, this preference rule specifies that
the most desirable cars are all the inexpensive cars,
the expensive cars with high capacity and safety,
the expensive cars with high capacity but low safety,
and the expensive cars with high safety but low capacity.

Given an ASO-rule $r$ of form (\ref{eq:ASO}),
we show how $r$ is encoded in a P-tree.
From the ASO-rule $r$, we build a P-tree $T_r$ in \figref{ASO_P},
where $\varphi_1 = \neg B \lor C_1 \lor (\bigwedge_{2 \leq i \leq m} \neg C_i)$,
$\varphi_i =C_i$ ($2 \leq i \leq m$),
and the dashed edge represents nodes labeled by the formulas $\varphi_3,\ldots,\varphi_{m-1}$
and every formula $\varphi_i$, $3 \leq i \leq m-1$, is constructed such that
the parent of $\varphi_i$ is $\varphi_{i-1}$, the left child of $\varphi_i$
is empty, and the right child of $\varphi_i$ is $\varphi_{i+1}$.

\begin{figure}
  \small
  \centering
	  \begin{tikzpicture}[->,>=stealth',
        level 1/.style={sibling distance=1.5cm, level distance=27pt},
        level 2/.style={sibling distance=1.5cm, level distance=27pt}
        ]
	    \node [main node,inner sep=1.7pt] (1){$\varphi_1$}
	      child {node [rectangle] (2) {} edge from parent[draw=none]}
	      child {node [main node,inner sep=1.7pt] (3) {$\varphi_2$}
	      	child {node [rectangle] (4) {} edge from parent[draw=none]}
					child {node [main node,inner sep=0.9pt] (5) {$\varphi_m$} [dashed]
						%child [solid] {node [rectangle,draw] (6) {}}
					}
				};
	  \end{tikzpicture}
  \caption{A P-tree $T_r$ ($T_P$)}
%\vspace{-0.2cm}
  \label{fig:ASO_P}
\end{figure}

\begin{thm}
\label{thm:ASO_P}
	Given an ASO-rule $r$, the P-tree $T_r$ has size
	linear in the size of $r$, and for every two outcomes $M$ and $M'$
	\begin{center}
		$M \succeq_r^{\tit{ASO}} M' \;\; \tit{iff} \;\; M \succeq_{T_r} M'$
	\end{center}
\end{thm}
\begin{proof}
	The P-tree $T_r$ induces a total preorder $\succeq_{T_r}$ where outcomes satisfying $\varphi_1$ are preferred
	to outcomes satisfying $\neg \varphi_1 \land \varphi_2$, which are then preferred to
	outcomes satisfying $\neg \varphi_1 \land \neg \varphi_2 \land \varphi_3$, and so on.
	The least preferred are the ones satisfying $\bigwedge_{1\leq i\leq m} \neg \varphi_i$.
	Clearly, this order $\succeq_{T_r}$ is precisely the order $\succeq^{\tit{ASO}}_r$
	given by the ASO rule $r$.
\end{proof}

%Clearly, the size of $T_r$ is linear in the size of the input $r$.
There are other ways of translating ASO-rules to P-trees. For instance,
it might be beneficial if the translation produced a more balanced tree.
We could proceed as in the proof of \propref{1}. %for $m+1$ clusters. 
%First, create the root node $N$ of $T^b_r$ and label it with the formula
%$\bigvee_{1 \leq i \leq \lfloor \frac{m+2}{2} \rfloor} \varphi_i$.
%Then, proceed recursively to construct $N$'s left subtree $T_1$ for 
%$\varphi_1,\ldots,\varphi_{\lfloor \frac{m+2}{2} \rfloor}$,
%and $N$'s right subtree $T_2$ for $\varphi_{\lfloor \frac{m+2}{2} \rfloor+1},
%\ldots, \varphi_{m+1}$.

For example, if $m=6$, we build the P-tree $T^b_r$ in \figref{PTree_tp},
where $\psi_1=\varphi_1 \vee \varphi_2 \vee \varphi_3$,
$\psi_2=\varphi_1$, $\psi_3=\varphi_2$, $\psi_4=\varphi_4$, and
$\psi_5=\varphi_5$.
The indices $i$'s of the formulas $\psi_i$'s indicate the order in which
the corresponding formulas are built recursively.


\begin{figure}[!ht]
	\centering
	  \begin{tikzpicture}[->,>=stealth',
	    level/.style={sibling distance=2cm/#1, level distance=27pt}]
	    \node [main node,inner sep=1.7pt] (1){$\psi_1$}
	      child {node [main node,inner sep=1.7pt] (2) {$\psi_2$}
	        child {node [rectangle] (3) {} edge from parent[draw=none]
	      		%child {node [rectangle,draw] (4) {}}
	      		%child {node [rectangle,draw] (5) {}}
					}
	        child {node [main node,inner sep=1.7pt] (6) {$\psi_3$}
	      		%child {node [rectangle,draw] (7) {}}
	      		%child {node [rectangle,draw] (8) {}}
					}
				}
	      child {node [main node,inner sep=1.7pt] (9) {$\psi_4$}
	        child {node [rectangle] (13) {} edge from parent[draw=none]}
	        child {node [main node,inner sep=1.7pt] (10) {$\psi_5$}
	      		%child {node [rectangle,draw] (11) {}}
	      		%child {node [rectangle,draw] (12) {}}
					}
	      };
	  \end{tikzpicture}
		\caption{$T^b_r$ when $m=6$}
%\vspace{-0.2cm}
  \label{fig:PTree_tp}
\end{figure}

This P-tree representation of a preference $r$ of the form (\ref{eq:ASO}) 
is balanced with height $\lceil \log_2 m \rceil$. Moreover, 
the property in \thmref{ASO_P} also holds for the balanced $T^b_r$
of size polynomial in the size of $r$.
In fact, the size of $T^b_r$ is in $O(s_r \log s_r)$, where $s_r$ is
the size of rule $r$.
It is clear that, though tree $T^b_r$ is larger than $T_r$ in size,
comparing outcomes could be done faster due to a smaller depth of $T^b_r$.

\paragraph{\bf Representing P-Trees as RASO-Theories.}

Preferences represented by compact P-trees cannot in general be
captured by ASO preferences without a significant (in some cases,
exponential) growth in the size of the representation. However, any 
P-tree can be represented as a set of \emph{ranked} ASO-rules, 
or an RASO-theory \cite{Brewka:ASO}, aggregated by the Pareto method.

%\nop{Introduce RASO and Pareto method}
%We first show how Pareto method is used to order outcomes with regard
%to a set of \emph{unranked} ASO-rules. Let $M$ and $M'$ be two outcomes.  
%Given a set $P$ of unranked ASO-rules, $M$ is weakly preferred to $M'$ 
%with respect to $P$, $M \succeq^u_P M'$, if $\SD_r(M) \leq \SD_r(M')$ 
%for every $r \in P$. Moreover, $M$ is strictly preferred to $M'$, $M 
%\succ^u_P M'$, if $M \succeq^u_P M'$ and $\SD_r(M) < \SD_r(M')$ for some 
%$r \in P$, and $M$ is equivalent to $M'$, $M \approx^u_P M'$,
%if $\SD_r(M) = \SD_r(M')$ for every $r \in P$.
%
%In general, the resulting preference relation is not total. However, by 
%ranking rules according to their importance in some cases, total preorders
%can be obtained. Let us assume $P=\{P_1,\ldots,P_g\}$ is a collection of 
%ranked ASO preferences divided into $g$ sets $P_i$, with each set $P_i$ 
%consisting 
%of ASO-rules of rank $d_i$ so that $d_1 < d_2<\ldots d_g$. We assume that 
%a lower rank of a preference
%rule indicates its higher importance.
%We define $M 
%\succeq^{rk}_P M'$ w.r.t $P$ if for every $i$, $1\leq i\leq g$, 
%$M \approx^u_{P_i} M'$, or if there exists a rank $i$ such that 
%$M \approx^u_{P_j} M'$ for every $j$, $j< i$, and $M \succ^u_{P_i} M'$.

Given a P-tree $T$, we construct an RASO-theory $\Phi_T$ as follows.
We start with $\Phi_T=\emptyset$.
For every node $t_i$ in a P-tree $T$, we update
$\Phi_T=\Phi_T \cup \{\varphi_i \overset{d_i}{\leftarrow} conditions\}$,
where $\varphi_i$ is the formula labeling node $t_i$,
$d_i$, rank of the ASO-rule, is the depth of node $t_i$,
and $conditions$ is the conjunction
of formulas $\varphi_j$ or $\neg \varphi_j$ labeling all nodes $t_j$
%such that $t_j \in \Inst(t_i)$.
that are ancestor nodes of $t_i$ in $T$ with two outgoing edges.
Whether $\varphi_j$ or $\neg \varphi_j$ is used depends on how the path 
from the root to
$t_i$ determines whether descending left ($\varphi_j$) or right ($\neg \varphi_j$)
at $t_j$.

For instance, the P-tree $T$ in \figref{LPT_PT_pt} gives rise to the following RASO-theory:

	\begin{framed}
%\vspace{-0.2cm}
		\noindent $\neg X_1 \overset{1}{\leftarrow}$.  \\%$\;\;\;\;\;$
		$X_3 \overset{2}{\leftarrow}$.\\
		$\neg X_2 \overset{3}{\leftarrow} X_3$.  $\;\;\;\;\;$
		$\neg X_4 \overset{3}{\leftarrow} \neg X_3$.\\
		$(X_2 \land X_4) \lor (\neg X_2 \land \neg X_4) \overset{4}{\leftarrow} X_3$. $\;\;\;\;\;$
		$X_2 \overset{4}{\leftarrow} \neg X_3$.
%\vspace{-0.2cm}
	\end{framed}

\begin{thm}
\label{thm:P_RASO}
	Given a P-tree $T$, there exists an RASO-theory $\Phi_T$ of size
	polynomial in the size of $T$ such that for every two outcomes $M$ and $M'$
	\begin{center}
		$M \succeq_{\Phi_T}^{\tit{RASO}} M' \;\; \tit{iff} \;\; M \succeq_{T} M'$
	\end{center}
\end{thm}
%\noindent Proof of \thmref{P_RASO} is omitted due to space constraint.
\begin{proof}
($\Leftarrow$) Let us assume $M \succeq_{T} M'$.
Denote by $(\varphi_{i_1},\ldots,\varphi_{i_j})$ the order of formulas 
labeling the path determined by $M$ from the root to a leaf.
Let $\varphi_{i_k}$, $1 \leq k \leq j$, be the first formula that
$M$ and $M'$ evaluate differently, in fact, $M \models \varphi_{i_k}$
and $M' \not \models \varphi_{i_k}$.
Denote by $d$ the depth of $\varphi_{i_k}$ in $T$.
Based on the construction of $\Phi_T$, 
for every RASO-rule $r$ of rank less than $d$, we have
$M \approx_{r}^{\tit{ASO}} M'$. For every RASO-rule $r$
of rank $d$, we have $M \succ_{r}^{\tit{ASO}} M'$ if
$r$ comes from $\varphi_{i_k}$; $M \approx_{r}^{\tit{ASO}} M'$
for other rules of rank $d$.
According to RASO ordering, $M \approx_{\Phi_T}^{\tit{RASO}} M'$ holds
if $\varphi_{i_k}$ does not exist; $M \succ_{\Phi_T}^{\tit{RASO}} M'$
holds, otherwise.  Therefore, $M \succeq_{\Phi_T}^{\tit{RASO}} M'$ holds.

($\Rightarrow$) Prove by contradiction.  We assume that $M \succeq_{\Phi_T}^{\tit{RASO}} M'$ 
and $M' \succ_{T} M$ hold.
We again denote by $(\varphi_{i_1},\ldots,\varphi_{i_j})$ the order of formulas
labeling the path determined by $M$ from the root to a leaf.
There must exist some formula $\varphi_{i_k}$, $1 \leq k \leq j$, such that
$M' \models \varphi_{i_k}$, $M \not \models \varphi_{i_k}$, and
all formulas $\varphi_\ell$, $1 \leq \ell \leq k-1$, are evaluated in the same way
by $M$ and $M'$. Based on RASO ordering, we have $M' \succ_{\Phi_T}^{\tit{RASO}} M$,
contradiction.
\end{proof}

Hence, the relationship between P-trees and ASO preferences can be summarized as follows.
Every ASO preference rule can be translated into a P-tree, and 
every P-tree into a theory of ranked ASO preference rules.
In both cases, the translations have size polynomial in the size of the input.
Examining the reverse direction, the size of the ASO rule translated from a P-tree
could be exponential, and the orders represented by ranked ASO theories
\emph{strictly include} the orders induced by P-trees as RASO-theories describe
\emph{partial} preorders in general.

\paragraph{\bf P-Trees Extend Possibilistic Logic.}

\nop{\comm{DEFINE THE SEMANTICS OF POSS LOG}}

%A possibilistic logic theory $\Pi$ over a vocabulary $\cI$ is a set of 
%\emph{preference pairs}
%\begin{center}
%	$\{ (\phi_1,a_1), \ldots, (\phi_m,a_m) \}$,
%\end{center}
%where every $\phi_i$ is a Boolean formula over $\cI$, and every $a_i$ is a real number
%such that $1\geq a_1>\ldots>a_m\geq 0$ (if two formulas have the same 
%importance level, they can be replaced by their conjunction).
%Intuitively, $a_i$ represents the importance of $\phi_i$, with larger values
%indicating higher importance.
%
%The \tit{tolerance degree} of outcome $M$ with regard to preference 
%pair $(\phi,a)$, $\TD_{(\phi,a)}(M)$, is defined by
%\[
% \TD_{(\phi,a)}(M) =
%  \begin{cases}
%   1, & M \models \phi \\
%   1-a, & M \not \models \phi
%  \end{cases}
%\]
%Based on that, the tolerance degree of outcome $M$ with regard to a \emph{set}
%$\Pi$ of preference pairs, $\TD_\Pi(M)$, is defined by 
%\begin{center}
%	$\TD_\Pi(M)=min\{\TD_{(\phi_i,a_i)}(M):1\leq i \leq m\}$.
%\end{center}
%The larger $\TD_\Pi(M)$, the more preferred $M$ is.
%
%For example, for the domain of vacations, we might have the following
%set of preference pairs $\{(\neg x_1 \wedge x_3,0.8), (x_2 \wedge x_4,0.5)\}$.
%According to the possibilistic logic interpretation, vacations satisfying 
%both preferences are the most preferred, those satisfying $\neg x_1 \wedge x_3$ 
%but falsifying $x_2 \wedge x_4$ are the next
%preferred, and those falsifying $\neg x_1 \wedge x_3$ are the worst.

Similarly as for ASO-rules, we can apply different methods to encode 
a possibilistic logic theories in P-trees. Here we discuss one of them.
We define $T_\Pi$ to be an unbalanced P-tree shown in \figref{ASO_P} with with 
labels $\varphi_i$ defined as follows: 
$\varphi_1=\bigwedge_{\substack{1\leq i \leq m}} \phi_i$, 
$\varphi_2=\bigwedge_{\substack{1\leq i \leq m-1}} \phi_i \wedge \neg \phi_m$,
$\varphi_3=\bigwedge_{\substack{1\leq i \leq m-2}} \phi_i \wedge \neg \phi_{m-1}$, and
$\varphi_m = \phi_1 \wedge \neg \phi_2$.
%$\varphi_{m+1}=\neg \phi_1$.

%\begin{figure}
%  \small
%  \centering
%	  \begin{tikzpicture}[->,>=stealth',
%	    level/.style={sibling distance=1.5cm/#1}]
%	    \node [main node] (1){$\varphi_1$}
%	      child {node [rectangle,draw] (2) {}
%				}
%	        child {node [main node] (3) {$\varphi_m$} [dashed]
%						child [solid] {node [rectangle,draw] (4) {}}
%					};
%	  \end{tikzpicture}
%  \caption{A P-tree $T_P$}
%  \label{fig:Poss_P}
%\end{figure}

\begin{thm}
\label{thm:Poss_P}
	Given a possibilistic theory $\Pi$, there exists a P-tree $T_\Pi$ of size
	polynomial in the size of $\Pi$ such that for every two outcomes $M$ and $M'$
	\begin{center}
		$M \succeq_\Pi^{\tit{Poss}} M' \;\; \tit{iff} \;\; M \succeq_{T_\Pi} M'$
	\end{center}
\end{thm}
\begin{proof}
	%Note that both induce in general total preorders of $m+1$ clusters.
	%It is clear that
	%outcome $M$ is in the $i$-th cluster induced by $\Pi$ if and only if
	%it is in the $i$-th cluster induced by $T_\Pi$.
	It is clear that the size of P-tree $T_\Pi$ is polynomial in the size of $\Pi$.
	Let $mi(M,\Pi)$ denote the maximal index $j$ such that $M$ satisfies
	all $\phi_1, \ldots, \phi_j$ in $\Pi$.
	(If $M$ falsifies all formulas in $\Pi$, we have $mi(M,\Pi)=0$.)
	We have that $M \succeq_\Pi^{\tit{Poss}} M'$ if and only if
	$mi(M,\Pi) \geq mi(M',\Pi)$, and $mi(M,\Pi) \geq mi(M',\Pi)$ if and only
	if $M \succeq_{T_\Pi} M'$.  Therefore, the theorem follows.
\end{proof}



\section{Reasoning Problems and Complexity}

In this section, we study decision problems on reasoning about preferences described as
P-trees, and provide computational complexity results for the three reasoning
problems defined below.

\begin{definition}
\label{def:dom}
  Dominance-testing ({\sc DomTest}): given a P-tree $T$ and two distinct outcomes
  $M$ and $M'$, decide whether $M \succeq_T M'$.
\end{definition}

\begin{definition}
\label{def:opt_test}
  Optimality-testing ({\sc OptTest}): given a P-tree $T$ and an outcome $M$ of $T$,
  decide whether $M$ is optimal.
\end{definition}

\begin{definition}
\label{def:opt_prop}
  Optimality-with-property ({\sc OptProp}): given a P-tree $T$ and some property $\alpha$ 
	expressed as a Boolean formula 
	over the vocabulary of $T$,
  decide whether there is an optimal outcome $M$ that satisfies $\alpha$.
\end{definition}

Our first result shows that P-trees support efficient dominance testing.
\begin{thm}
\label{thm:dom}
	The {\sc DomTest} problem can be solved in time linear in the
	height of the P-tree $T$.
\end{thm}
\begin{proof}
	The {\sc DomTest} problem can be solved by walking down the tree.
	The preference between $M$ and $M'$ is determined at the first non-leaf node
	$n$ where $M$ and $M'$ evaluate $\varphi_n$ differently.  If such
	node does not exist before arriving at a leaf, $M \approx_T M'$.
\end{proof}
An interesting reasoning problem not mentioned above is to decide whether 
there exists an optimal outcome with respect to the order given by a P-tree.
However, this problem is trivial as the answer simply depends on whether 
there is any outcome at all. However, optimality \emph{testing} is a 
different matter. Namely, we have the following result.

\begin{thm}
\label{thm:opt_test}
	The {\sc OptTest} problem is coNP-complete.
\end{thm}
\begin{proof}
%	Need to show that deciding whether the given outcome $M$ is \tit{not} an optimal 
%	outcome in a given P-tree $T$ is NP-complete.
%	This complement problem is  in class NP because one can guess an outcome $M'$ in
%	polynomial time and verify in polynomial time that $M' \succ_T M$.
%	Hardness follows from a polynomial time reduction from SAT \cite{Garey:1979}.
%	Details of the reduction is omitted due to limited space.
%	(Membership) The problem is in class NP because one can guess an outcome $M'$ in
%	polynomial time and verify in polynomial time that $M' \succ M$.
%
%	(Hardness) The hardness follows from a polynomial time reduction from SAT \cite{Garey:1979}.
We show that the complementary problem, testing non-optimality of an outcome
$M$, is NP-complete. Membership is obvious. A witness of non-optimality of $M$
is any outcome $M'$ such that $M' \succ_T M$, a property that can be verified 
in linear time (cf. Theorem \ref{thm:dom}).
	NP-hardness follows from a polynomial time reduction from SAT \cite{Garey:1979}.
	Given a CNF formula $\Phi=c_1\land\ldots\land c_n$ over a set of variables
	$V=\{X_1,\ldots,X_m\}$, we construct a P-tree $T$ and an outcome $M$
as follows.
\begin{enumerate}
\item We choose $X_1,\ldots,X_m,unsat$ as attributes, where $unsat$ is 
a new variable;
\item we define the P-tree $T_\Phi$ (cf. \figref{P_opt_2_comp}) to consist of a 
single node labeled by $\Psi=\Phi \land \neg unsat$;
\item we set $M=\{unsat\}$.
\end{enumerate}

	We show that $M=\{unsat\}$ is not an optimal outcome if and only if
	$\Phi=\{c_1,\ldots,c_n\}$ is satisfiable.

\noindent
	($\Rightarrow$) Assume that $M=\{unsat\}$ is not an optimal outcome.
	Since $M \not \models \Psi$, $M$ belongs to the right
	leaf and there must exist an outcome $M'$ such that $M' \succ M$.
	This means that $M' \models \Phi \wedge \neg unsat$. Thus, 
	$\Phi$ is satisfiable.

\noindent
	($\Leftarrow$) Let $M'$ be a satisfying assignment to $\Phi$ over $\{X_1,\ldots,X_m\}$.
	Since no $c_i \in \Phi$ mentions $unsat$, we can assume $unsat \not \in M'$.
	So $M' \models \Psi$ and $M'$ is optimal.
	Thus, $M=\{unsat\}$ is not optimal.
\end{proof}

\begin{figure}
  \centering
	  \begin{tikzpicture}[->,>=stealth',
	    level/.style={sibling distance=1.5cm/#1, level distance=30pt}]
	    \node [main node,inner sep=4pt] (1){$\Psi$}
	      %child {node [rectangle,draw] (2) {}}
				;
	  \end{tikzpicture}
  \caption{The P-tree $T_\Phi$}
%\vspace{-0.2cm}
  \label{fig:P_opt_2_comp}
\end{figure}


\begin{thm}
\label{thm:opt_prop}
	The {\sc OptProp} problem is $\deltap{2}$-complete.
\end{thm}
\begin{proof}
	(Membership) The problem is in the class $\deltap{2}$. Let $T$ be a given
	preference tree. To check whether there is an optimal outcome that 
	satisfies a property $\alpha$, we start at the root of $T$ and move down.
	As we do so, we maintain the information about the path we took by 
	updating a formula $\psi$, which initially is set to $\top$ (a generic
	tautology). Each time we move down to the left from a node $t$, we 
	update $\psi$ to $\psi\land\vph_t$, and when we move down to the right,
	to $\psi\land\neg\vph_t$. To decide whether to move down left or right 
	form a node $t$, we check if $\vph_{t} \land \psi$ is satisfiable by 
	making a call to an NP oracle for deciding satisfiability. If $\vph_{t} 
	\land \psi$ is satisfiable, we proceed to the left subtree and, 
	otherwise, to the right one. We then update $t$ to be the node we moved 
	to and repeat. When we reach a leaf of the tree (which represents a
	cluster of outcomes), this cluster is non-empty, consists of all 
	outcomes satisfying $\psi$ and all these outcomes are optimal. Thus, 
	returning YES, if $\psi\land \alpha$ is satisfiable and NO, otherwise, 
	correctly decides the problem. Since the number of oracle calls is 
	polynomial in the size of the tree $T$, the problem is in the class
	$\deltap{2}$. 

\medskip
\noindent
	(Hardness) The maximum satisfying assignment (MSA) problem\footnote{
		Given a Boolean formula $\Phi$ over $\{X_1,\ldots,X_n\}$, the
		maximum satisfying assignment (MSA) problem is to decide whether $x_n=1$ 
		in the lexicographically maximum satisfying assignment for $\Phi$.
		(If $\Phi$ is unsatisfiable, the answer is $\no$.)
	}
	\cite{Krentel:88} is $\deltap{2}$-complete.
	We first show that MSA remains $\deltap{2}$-hard if we
	restrict the input to Boolean formulas that are satisfiable and 
        have models other than
	the all-false model (i.e., $0_1\ldots 0_n$).

	\begin{lem}
	\label{lem:MSA_sat}
		The MSA problem is $\deltap{2}$-complete when $\Phi$ is satisfiable and has models
		other than the all-false model.
	\end{lem}
	\begin{proof}
		Given a Boolean formula $\Phi$ over $\{X_1,\ldots,X_n\}$, we define
		$\Psi=\Phi \lor (X_0\land\neg X_1 \land\ldots \land\neg X_n)$ over
		$\{X_0,X_1,\ldots,X_n\}$.
		It is clear that $\Psi$ is satisfiable, and has at least one model other than
		the all-false one.
		Let $M$ be a lexicographically maximum assignment satisfying $\Phi$ and
		$M$ has $X_n=1$.
		Extending $M$ by $X_0=1$ yields a lexicographically maximum assignment 
		satisfying $\Psi$ and this assignment satisfies $X_n=1$.
		Conversely, if $M$ is a lexicographically maximum assignment satisfying $\Psi$
		and $X_n=1$ holds in $M$, it follows that $M \models \Phi$.  Thus,
		restricted $M$ to $\{X_1,\ldots,X_n\}$, the assignment is lexicographically maximal
		satisfying $\Phi$.
	\end{proof}

	We now show the hardness of the {\sc OptProp} problem by a reduction from this
	restricted version of the MSA problem. Let $\Phi$
	be a satisfiable propositional formula over variables $X_1,\ldots,X_n$ 
	that has at least one model other than the all-false one.
	We construct an instance of the {\sc OptProp} problem as follows.
	We define the P-tree $T_\Phi$ as shown in \figref{P_opt_3_comp}, 
where every node is labeled by formula $\Phi \land X_i$, and we set
	$\alpha=X_n$.

	\begin{figure}
	  \small
	  \centering
		  \begin{tikzpicture}[->,>=stealth',
		    level/.style={sibling distance=2.5cm/#1, level distance=40pt}]
		    \node [main node,inner sep=0pt] (1){$\Phi \! \land \! x_1$}
		      child {node [main node,inner sep=0pt] (2) {$\Phi \! \land \! x_n$} [dashed]
		      	%child [solid] {node [rectangle,draw] (3) {}}
		      };
		  \end{tikzpicture}
	  \caption{The P-tree $T_\Phi$}
%\vspace{-0.2cm}
	  \label{fig:P_opt_3_comp}
	\end{figure}

	Our P-tree $T_\Phi$ induces a total preorder consisting of a sequence of
	singleton clusters, each containing an outcome satisfying $\Phi$,
	followed by a single cluster comprising all outcomes that falsify
	$\Phi$ and the all-false model.
	By our assumption on $\Phi$, the total preorder has at least 
        two non-empty clusters.
	Moreover, all singleton clusters
	preceding the last one are ordered lexicographically. Thus, the optimal 
	outcome of $T_\Phi$ satisfies $\alpha$ if and only if the lexicographical
	maximum satisfying outcome of $\Phi$ satisfies $x_n$.
\end{proof}

%\section{Aggregating P-Trees}
%In the setting of multi-agent systems when agents that collaborate
%with each other need to make a common decision, it is important to
%aggregate their individual preferences in order to reach a collective agreement.
%We adopt two approaches to solve aggregation problems of P-trees:
%the Pareto method and voting rules.
%
%\subsection{The Pareto method}
%%When an agent express her preferences in terms of a set of P-trees, it is important to consider these P-trees
%%together to compute optimal outcomes to support further decision making process.
%%For instance, in a recommendation system preferences are aggregated so that the top $K$ outcomes are computed.
%We define the Pareto method problem and present its implementation using answer set programming (ASP) 
%\cite{mt99:stable}.
%
%
%\begin{definition}
%\label{def:opt_pareto}
%  Optimality-testing-Pareto ({\sc OptTestPa}): given a set $P$ of P-trees
%	and an outcome $M$,
%	decide whether $M$ is a Pareto optimal outcome with regard to $P$.
%\end{definition}
%
%\begin{definition}
%\label{def:dom_pareto}
%  Dominance-testing-Pareto ({\sc DomTestPa}): given a set $P$ of P-trees
%	and an outcome $M$,
%	decide whether there exists another outcome $M'$ such that $M' \succ_P M$,
%	that is, $M$ is Pareto dominated by $M'$ with respect to $P$.
%\end{definition}
%
%The {\sc OptTestPa} problem is \tit{coNP-complete}, 
%and the {\sc DomTestPa} problem is \tit{NP-complete}.
%It is clear that the {\sc OptTestPa} problem is in coNP.
%Hardness of {\sc OptTestPa} follows from \thmref{opt_test} where a single P-tree is considered.
%As for the {\sc DomTestPa} problem, obviously it is in NP, and its hardness is due to
%the fact that its solution leads to the solution of the complement of {\sc OptTestPa}.
%
%The implementation is based on a testing program in ASP that takes an outcome as input and computes
%a strictly better one if such an outcome exists.
%The computation starts with an arbitrary outcome which is given to the testing program.
%If the testing program fails to generate a strictly better outcome, the input outcome is Pareto
%optimal with regard to $P$.  If the testing program generates such an outcome that is strictly
%better, the new outcome becomes the input and the process repeats until Pareto optimality is reached.
%
%\smallskip \noindent \textbf{Encoding P-trees as logic programming rules.}
%A P-tree can be encoded as a set of logic programming rules.
%To solve the {\sc OptTestPa} problem, we need to encode P-trees in such a way that
%the dominance between two outcomes can be determined.
%
%Let $\varphi_i$ be the formula that labels a node $t$ in a P-tree $T$ at depth $d^T_i$,
%$\Inst(t)=\{t_1,\ldots, t_j\}$ where each $t_q$, $1\leq q \leq j$, 
%is labeled by a formula $\varphi_{i_q}$.
%The location of $t$ is determined by its
%depth $d^T_i$ and by how the formulas $\varphi_{i_1},\ldots,\varphi_{i_j}$
%labeling $\Inst(t)$ are evaluated (they determine whether we descend to the left or
%to the right child as we descend down the tree).
%Thus, the preferences with $\varphi_i$ can be represented by the following
%logic programming rule.
%\begin{equation} \label{eq:prefASP}
%\begin{split}
%  \small
%  pt&(T,d^T_i,i) \; \ifLparse \; \form(T,i_1), \ldots, \form(T,i_k), \\
%    &not \; \form(T,i_{k+1}), \ldots, not \; \form(T,i_j), \form(T,i).
%\end{split}
%  \end{equation}
%In rule (\eqtref{prefASP}), predicate $\form(T,\ell)$ holds for an outcome $M$ 
%if in P-tree $T$ outcome $M$ satisfies
%formula $\varphi_\ell$, and predicate $pt(T,d^T_i,i)$ holds for $M$ if
%in $T$ outcome $M$ satisfies formula $\varphi_i$ at depth $d^T_i$.
%The rule indicates that $pt(T,d^T_i,i)$ holds for $M$
%if in $T$ outcome $M$ satisfies formulas $\varphi_{i_1},\ldots,\varphi_{i_k} \in \Inst(t)$,
%falsifies formulas $\varphi_{i_{k+1}},\ldots,\varphi_{i_j} \in \Inst(t)$,
%and satisfies formula  $\varphi_i$.
%
%Let us recall the example P-tree $T$ in \figref{PTree_compact} and represent the P-tree as
%a set of logic rules in \figref{P_ASP}, where predicate $outcome(X_i,a)$ holds
%for outcome $M$ if $M$ assigns value $a$ to variable $X_i$.
%
%\begin{figure}[ht!]
%	\begin{framed}
%\footnotesize
%		\begin{verbatim}
%1  id(1).
%2  form(1,1) :- outcome(1,1).
%3  form(1,2) :- outcome(2,1), outcome(3,1).
%4  form(1,3) :- outcome(4,0).
%5  form(1,4) :- outcome(2,0), outcome(4,1).
%6  pt(1,1,1) :- form(1,1).
%7  pt(1,2,2) :- form(1,1), form(1,2).
%8  pt(1,2,4) :- not form(1,1), form(1,4).
%9  pt(1,3,3) :- form(1,1), form(1,3).
%		\end{verbatim}
%	\end{framed}
%	\caption{Encode $T$ in ASP}
%  \label{fig:P_ASP}
%\end{figure}
%
%
%\smallskip \noindent \textbf{The $P$-dominance-Pareto problem in ASP.}
%We assume the starting outcome $M$ is $\vec{0}$, that is, an outcome assigning
%0 to all variables. The testing program $P$ is built by adding the following rules
%to the encoding of $P$:
%\begin{enumerate}
%	\item For each formula $\varphi_i \in P$ such that $M \models \varphi_i$, include a fact 
%				$pt\_start(T,d^T_i,i)$ where $T$ is the id of the P-tree that 
%				$\varphi_i$ comes from, and $d^T_i$ is the depth of $\varphi_i$ in $T$.
%	\item Include the rules in \figref{P_dom_encod}.
%\end{enumerate}
%
%\begin{figure}[ht!]
%	\begin{framed}
%\footnotesize
%		\begin{verbatim}
% 1  1{ outcome(X,Y) : val(Y) }1 :- var(X).
% 
% 2  split_depth(ID,D) :- pt(ID,D,F), 
%      not pt_start(ID,D,F).
% 3  split_depth(ID,D) :- not pt(ID,D,F), 
%      pt_start(ID,D,F).
% 4  first_split_depth(ID,D) :- id(ID), 
%      D=#min[split_depth(ID,Depth)=Depth].
% 
% 5  eq(ID) :- id(ID), 
%      first_split_depth(ID,#supremum).
% 6  better(ID) :- first_split_depth(ID,D), 
%      D != #supremum, pt(ID,D,F).
% 7  bettereq(ID) :- eq(ID).
% 8  bettereq(ID) :- better(ID).
% 
% 9  :- not exist_better.
%10  exist_better :- better(ID).
%11  :- id(ID), not bettereq(ID).
%		\end{verbatim}
%	\end{framed}
%	\caption{Encode the {\sc DomTestPa} problem in ASP}
%  \label{fig:P_dom_encod}
%\end{figure}
%
%
%
%\subsection{Voting rules}
%As P-trees are in general total preorders over outcomes,
%we consider voting rules to aggregate them, such as positional scoring rules
%(Plurality, $k$-Approval, Borda, etc), Maximin, and Copeland.
%
%\tc{
%	Votes with empty leaves make a difference when we apply positional scoring
%	rules on them, whereas they do not matter for comparison-based voting rules.
%	So for positional scoring rules, do we consider votes with empty leaves valid or invalid?
%	If invalid, we need to identify those votes and discard them from the profile, but
%	checking if one P-tree is invalid takes exponentially many calls of the SAT oracle
%	in the worst case!
%	If valid, we still need to compute, for every P-tree, which leaves are
%	empty, which again takes exponentially many calls to a SAT solver!
%	However, I think that one can assume all votes are carefully specified
%	so that empty leaf does not exist.
%}
%
%\smallskip \noindent \textbf{Plurality.}
%Assume for vote (P-tree) $T$ there are $c$ outcomes with rank 1.
%If $c=0$, it means that $T$ vetos all outcomes, thus $T$ is an invalid
%vote and is removed from the profile.
%Otherwise, every top-ranked outcome gets score $\frac{1}{c}$.
%Understanding the Plurality this way, we need to know the number of top-ranked
%outcomes in every P-tree.  The problem of ``counting the number of
%outcomes that satisfy a formula $\Phi$" is \#P-hard.
%
%\smallskip \noindent \textbf{Copeland.}
%As proved by Lang et al \cite{lang:aggLP}, for LP-trees in classes $\{UI,CI\} \times \{UP,CP\}$,
%computing the Copeland score of a given outcome in a given profile is \#P-complete,
%and the Maximin evaluation problem is coNP-complete.
%These results should still hold in the setting of P-trees which extend the
%framework of LP-trees.


\section{Conclusions}

We investigated the qualitative preference representation language of 
\tit{preference trees}, or \tit{P-trees}. This language was 
introduced in early 1990s (cf. \cite{fraser1993,fraser1994}), but have 
not received a substantial attention as a formalism for preference 
representation in AI. We studied formally the attribute of compact 
representations of
P-trees, established its relationship to other preference languages
such as lexicographic preference trees, possibilistic logic and
answer-set optimization. For several preference reasoning problems 
on P-trees we derived their their computational complexity.

P-trees are quite closely related to possibilistic logic theories or
preference expressions in answer-set optimization. However, they allow
for much more structure among formulas appearing in these latter two
formalisms (arbitrary trees as opposed to the linear structure of 
preference formulas in the other two formalisms). This structure allows 
for representations of conditional preferences. P-trees are also more
expressive than lexicographic preference trees. This is the case even 
for P-trees in which every node is labeled with a formula involving just
two attributes, as we illustrated with the $00\succ 11\succ 01\succ 01$ 
example. Such P-trees are still simple enough to correspond well to
the way humans formulate hierarchical models of preferences, with all
their decision conditions typically restricted to one or two attributes. 

Our work shows that P-trees form a rich preference formalism that
deserves further studies. Among the open problems of interest are 
those of learning P-trees and their compact representations, aggregating
P-trees coming from different sources (agents), and computing
optimal consensus outcomes. These problems will be considered in the 
future work. 
