In my Ph.D. dissertation, I have studied problems arising in various aspects of preferences:
preference modeling, preference learning, and preference reasoning, 
when preferences concern outcomes ranging over combinatorial domains.
Preferences is a major research component
in artificial intelligence (AI) and decision theory, and is closely related to the 
social choice theory considered by economists and political scientists. 
In my dissertation, I have exploited emerging connections between 
preferences in AI and social choice theory. 
Most of my research is on qualitative preference representations that extend and combine
existing formalisms such as conditional preference nets, 
lexicographic preference trees, answer-set optimization programs,
possibilistic logic, and conditional preference networks; on 
learning problems that aim at discovering qualitative preference models and 
predictive preference information from practical data; and on
preference reasoning problems centered around qualitative preference optimization 
and aggregation methods.
Applications of my
research include recommender systems, decision support tools, multi-agent systems,
and Internet trading and marketing platforms.
