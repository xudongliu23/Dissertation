For my PhD dissertation I study problems in preference representation, 
reasoning and learning, when preferences concern outcomes ranging over
combinatorial domains.
Preferences is a major research component
in artificial intelligence (AI) and decision theory, and is closely related to the 
social choice theory considered by economists and political scientists. In my research I 
will exploit emerging connections between preference reasoning in AI and social choice. 
Most of my research is on qualitative preference representations that extend and combine
existing formalisms such as conditional preference nets, 
lexicographic preference trees and answer-set optimization programs; on 
learning problems that aim at discovering qualitative preference modelss and 
predictive preference information from practical data; and on
preference reasoning problems centered around qualitative preference optimization 
and aggregation methods.

